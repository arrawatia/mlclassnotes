\documentclass[a4, 12pt, english, USenglish]{scrreprt}
% \usepackage{venn}
\usepackage[latin1]{inputenc}
\usepackage{makeidx}
% \usepackage{pdftricks}
\usepackage{graphicx}
% \usepackage[final]{pdfpages}

\usepackage{geometry, upgreek, booktabs, babel}
\usepackage[journal=rsc,xspace=true]{chemstyle}
\usepackage[version=3]{mhchem}
% \usepackage[footnotes]{notes2bib}
\usepackage[final]{microtype}
\usepackage[final, inactive]{pst-pdf}
\usepackage[colorlinks]{hyperref}

% equals with a "set" on top
\newcommand{\defeq}{\ensuremath{\stackrel{\mbox{set}{=}}}}


\renewcommand{\topfraction}{.85}
\renewcommand{\bottomfraction}{.7}
\renewcommand{\textfraction}{.15}
\renewcommand{\floatpagefraction}{.66}
\renewcommand{\dbltopfraction}{.66}
\renewcommand{\dblfloatpagefraction}{.66}
\setcounter{topnumber}{9}
\setcounter{bottomnumber}{9}
\setcounter{totalnumber}{20}
\setcounter{dbltopnumber}{9}


\newcommand{\xscreenshot}[2]{
\begin{figure}[htb]
\begin{center}
\em Missing imagefile file #1
\end{center}
\label{#1}
\caption{#2}
\end{figure}}

\newcommand{\zcreenshot}[3]{
\begin{figure}[htb]
\includegraphics[width=#3]{screenshots/#1.jpg}
\label{#1}
\caption{#2}
\end{figure}}

\newcommand{\screenshot}[2]{
\begin{figure}[htb]
\includegraphics[width=150mm]{screenshots/#1.jpg}
\label{#1}
\caption{#2}
\end{figure}}





\newcommand{\sscreenshot}[3]{
\begin{figure}[htb]
\includegraphics[width=7500mm]{screenshots/#1.jpg}
\label{#2}
\caption{#3}
\end{figure}}

% XXX Should put a little arrow above its parameter.
\newcommand{\vectorXX}[1]{\ensuremath{#1}}

\newcommand{\fApartial}[1]{\ensuremath{\frac{\partial f}{\partial A_{#1}}}}
\newcommand{\jpartial}[1]{\ensuremath{\frac{\partial J}{\partial \theta_{#1}}}}
\newcommand{\thetaipartial}{\ensuremath{\frac{\partial}{\partial{\theta_i}}}}
\newcommand{\thetapartial}{\ensuremath{\frac{\partial}{\partial{\theta}}}}
\newcommand{\half}{\ensuremath{\frac{1}{2}}}
\newcommand{\sumim}{\ensuremath{\sum_{i=1}^{m}}}
\newcommand{\intinf}{\ensuremath{\int_{-\infty}^{\infty}}}
\newcommand{\Ft}{\ensuremath{{\cal{F}}}}
\newcommand{\ft}[1]{\ensuremath{{\cal{F}}({#1})}}

\newcommand{\sinc}[1]{\ensuremath{\mbox{sinc}{#1}}}


\newcommand{\bb}[1]{\ensuremath{{\bf{#1}}}} % XXX Should be blackboard bold
\newcommand{\proj}[2]{\ensuremath{{\bb {#1}}_{#2}}}
\newcommand{\braces}[1]{\ensuremath{\left\{{#1}\right\}}}
\newcommand{\brackets}[1]{\ensuremath{\left[{#1}\right]}}
\newcommand{\parens}[1]{\ensuremath{\left({#1}\right)}}
\newcommand{\absval}[1]{\ensuremath{\left|{#1}\right|}}
\newcommand{\sqbraces}[1]{\ensuremath{\left[{#1}\right]}}
\newcommand{\commutator}[2]{\sqbraces{{#1}, {#2}}}


\newcommand{\dyad}[1]{\ensuremath{\ket{{#1}}\bra{{#1}}}}
\newcommand{\trace}[1]{\ensuremath{\mbox{tr}\, {#1} }}
\newcommand{\erf}[1]{\mbox{erf}\left(#1\right)}
\newcommand{\erfc}[1]{\mbox{erfc}\left(#1\right)}
\newcommand{\mXXX}[1]{\marginpar{\tiny{\bf Rmz:} {\it #1}}}
\newcommand{\celcius}{\ensuremath{^\circ}C}

\newcommand{\ev}[1]{\ensuremath{\left\langle{}#1{}\right\rangle}}
\newcommand{\ket}[1]{\ensuremath{\mid{}#1{}\rangle}}
\newcommand{\bra}[1]{\ensuremath{\langle{}#1{}\mid}}
\newcommand{\braKet}[2]{\ensuremath{\left\langle{}#1{}\mid{#2}\right\rangle}}
\newcommand{\BraKet}[3]{\ensuremath{\left\langle{}#1{}\mid{#2}\mid{#3}\right\rangle}}
\newcommand{\evolvesto}[2]{\ensuremath{{#1}\mapsto{#2}}}
\newcommand{\inrange}[3]{\ensuremath{{#1} \in \braces{{#2}, \ldots,{#3}}}}

\newenvironment{wikipedia}[1]
{
 {\bf From wikipedia: {\it #1}}
 \begin{quote}
}
{
 \end{quote}
}

\newcommand{\idx}[1]{{\em #1}\index{#1}}
\newcommand{\idX}[1]{{#1}\index{#1}}

\usepackage{url}
\newcommand{\tm}{\ensuremath{^{\mbox{tm}}}}
\newcommand{\aangstrom}{\AA{}ngstr\"{o}m{}\ }
%\newcommand{\aaunit}{\mbox{\AA}} % Just use A with ring, once encoding works properly
\newcommand{\aaunit}{\angstrom} % Just use A with ring, once encoding works properly
\newcommand{\munchen}{M\"unchen}
\newcommand{\zurich}{Z\"urich}
\newcommand{\schrodinger}{Schr\"odinger}
\newcommand{\ReneJustHauy}{Ren\'e-Just Ha\"uy}

%% Lavousier (with a lot fo weird spelling)


%% Crystallographic notation
%Coordinate
\newcommand{\crCoord}[3]{\mbox{\(#1,#2,#3\)}}
%Direction
\newcommand{\crDir}[3]{\mbox{\(\left[#1 #2 #3\right]\)}}
%Family of directions
\newcommand{\crDirfam}[3]{\mbox{\(\left<{}#1 #2 #3\right>\)}}
%Plane
\newcommand{\crPlane}[3]{\mbox{\(\left(#1 #2 #3\right)\)}}
%Family of planes
\newcommand{\crPlanefam}[3]{\left\{#1 #2 #3\right\}}

\newcommand{\oneCol}[2]{
  \ensuremath{\left(\begin{array}{r}{#1}\\{#2}\end{array}\right)}
}

\newcommand{\twoCol}[4]{
  \ensuremath{\left(\begin{array}{rr}{#1}&{#2}\\{#3}&{#4}\end{array}\right)}
}

%Negative number
\newcommand{\crNeg}[1]{\bar{#1}}

\makeindex

\begin{document}

\title{Lecture notes from the \\
online  machine learning\\
tought by Andrew Ng \\
fall 2011}

\author{Bj\o{}rn Remseth \\ rmz@rmz.no}

\maketitle
\tableofcontents

% Comment out this in final version!

% \parskip=\bigskipamount
% \parindent=0pt.

\begin{abstract}

\end{abstract}

\chapter*{Introduction}

These are my notes for the course in machine learning  based on Andrew
Ng's lectures, the autumn 2011.    

I usually watched the videos while typing notes in \LaTeX.  I have
experimented with various note-taking techniques including free text,
mindmaps and handwritten notes, but I've ended up using \LaTeX, since
it's not too hard, it gives great readability for the math that
inevitably pops up in the things I like to take notes about, and it's
easy to include various types of graphics.  The graphics in this video
is exclusively screenshots copied directly out of the videos, and to a
large extent, but not  completely, the text is based on Ng's
narrative.   I haven't been very creative, that wasn't my purpose.  I
did take more screenshots than are actually available in this text.
Some of them are indicated in figures stating that a screenshot is
missing.  I may or may not get back to putting these missing
screenshots back in, but for  now the are just not there.  Deal with
it .-)

This document will every now and then be made available on
\url{http://dl.dropbox.com/u/187726/machine-learning-notes.pdf}.   The
source code can be cloned on git on \url{https://github.com/la3lma/mlclassnotes}.


A word of warning: These are just my notes.  They should't be
interpreted as anything else.  I take notes as an aid for myself.
When I take notes I find myself spending more time with the subject at
hand, and that alone lets me remember it better.  I can also refer to
the notes, and since I've written them myself, I usually find them
quite useful.   I state this clearly since the use of \LaTeX\ will
give some typographical cues that may lead the unwary reader to
believe that this is a textbook or something more ambitious.  It's
not.  This is a learning tool for me.  If anyone else reads this and
find it useful, that's nice. I'm happy,  for you, but I didn't have
that, or you in mind when writing this.   That said, if you have any
suggestions to make the text or presentation better, please let me
know.  My email address is la3lma@gmail.com.

\chapter{Linear regression}

Normal Equation Noninvertability.:

ocateve: pinv(X'*X) * X' * y  is the normal equation \(X^TX)^{-1} X' y\) implemented Octaves pseudoinverse function.

If there are linearly depenent features (e.g. feet/meter features, that are linear equtions)
If there are too many features. Could cause noninvertability.  Deleting features or using
\idx{regularization} could do the trick.


If X'X singular then look for redundant features, then delete one of them.  Otherwise nuke some features or regularization.


\chapter{Octave:}

Prototyping in Octave is very efficient.  Only for very large scale
implementations do we need lower leve language implementations.
Programmer time is incredibly valuable so optimizing that is the first
optimization that should always be done.

Prototyping langues: numPy, R, Octave, Matlab, Python.   ALl of them slightly clunky than octave.

\begin{verbatim}
Setting the prompt "PS1('>> ')".    A==B => a matrix of  results of the logical thing.

Printing:
    disp(sprintf(' 2 decimals: %0.2f', a))

format long -> print with lots of digits
format short -> print with few digits

v = 1:0.1:2   -> all elements from one to two in steps 0.1 (default stepsize 1)

ones(2,3)  -> 2x3 matrix of only ones.
zeros(1,3) -> 1x3 matrix of zeros
rand(3,3) -> 3x3 andom numbers (uniform)
randn(3,3) -> 3x3 random numbers (gaussian distribution)

hist(w) is a  histogram function.  Really nice.

eye(4) -> 4x4 identity matrix (diagonal)

help(eye) 

the "size" command returns a 1x2 matrix that is the size of a matrix.

size(A,1) gives the first element (rows), 2 gives column.
length(A) gives the length of the longest dimension.  Usually we use this only for vectors.

pwd shows current directory
cd changes directory
ls lists files on desktop

load('feturesX.dat') loads files with space separated rows and sets a variable from the filename

the "who" command shows which matrics are available.

whos shows the sizes etc. for the variables.
the clear command removes variable.
saving data.

v=priceY(1:10) the first ten elements of priceY.

save hallo.mat v; 

saves the variable v into the file hallo.mat

clear without parameters will remove everything.

save hello.txt  v --ascii 

will save as a text (not binary which is the default)

A(3,2) get the A_3^{(2)} element
A(2,:) gets the second row 

A([1 3], :) get everything from the first and third rows.

stopped at 8:30
\end{verbatim}




\chapter{Logistic regression: Classification}

The variable y we want to classify into classes: Spam/not spam.
Fraudulent(not raudulent online transactions. Classifying tumors
malign/benign.

In all these cases \(y \in \braces{0, 1}\).  The zero being called the
\idx{negative class} and the one being called the \idx{positive
class}.   The assignemtn of the two classes to positive and negative
class is somewhat arbitrary, but the negative is often the absence of
something and the positive presence.

We'll start with a classification problem with only two possible
values, called \idx{two-class} or \idx{binary} classification problem.
Late we'll also look at \idx{multiclass} classification problems.

How do we develop a classification problem.  One thing could do is to
use linear regression

\screenshot{linear-regression-classifier}{Linear regression classifier}

and then threshold the hypothesis at some value e.g. 0.5.    We then
have a classifier algorithm.   In the exmple above this looks
reasonable, but adding an outlier to the right will tilt the
regression line  and that makes the prediction very bad as shown in
the example \ref{tilted-regression}.

\screenshot{tilted-regression}{Tilted regression classifier}

Applying linear regression to a classification problem usually isn't a
good example.   Also, \(h(x)\) can output values much larger than 1 or
smaller than 0, and that seems ind of weird.

Logistic regression is nicer than that, among other things since \(0
\leq h_\theta \leq 1 \).  The term regression in ``logistic
regression''is the name the algorithm was given for historical
reasons, even though it is really a classification algorithm.

The \idx{hypothesis representation} in logistic regression is:

\[
   h(x) = g(\theta^T x) = \frac{1}{1 + e^{\theta^T x}}
\]

\(g\) is called the \idx{sigmoid function} or the \idx{logistic
function}, and the latter is what gives logistic regression its name.
The two names are synonyms and can be used interchangably (is that a
word?.

\screenshot{sigmoidfunction}{Sigmoid function}

What we need to do is to fit the parameters to the \(\theta\), and
we'll get an algorithm for this soon enough.

\screenshot{logisticinterpretation}{Interpretation of results from
  logistic regression}

\subsection*{Decision boundary}

More Intuition for the hypothesis function for logistic regression (24
secs into the video).

When should we predict one and when should we predict zero?   One way
is just to select a threshold. 

\zcreenshot{sigmoid-selection-criterion}{Sigmoid selection criterion}{7cm}


In essence, our choice will flip when \(\theta^T \geq 0 \).

We can use this fact to better unnerstand how logistic regression
makes decisions.    Assume \(h_\theta(x) = g(\theta_0 + \theta_1 x_1 +
\theta_2 x_2) \).  We'll lok at how to fit the parameters later, but
let's just assume that we've done so  and have \(\theta =
\braces{-3,1,1}^T\).

The trick is to find the line that most with the smallest wrong
classification (error) divides the classes in the feature plane.

\screenshot{boundaryplot}{The decision boundary for a linear
  regression classifier}


The dividing line is called the \idx{decision boundary}.  The decision
boundary is the property of the parameters, not the data set.   We can
use the data to determine the theta, but after that we don't need the
training set.


\subsection*{Non-linear boundaries}

We can add higher order polynomials to model non-linear features.


\screenshot{circularboundary}{A circular decision boundary}

The boundary is again  property of the hypothesis function.

We can use even higher order polynomials.   The boundaries can be then
be very complex. Ellipsis, strange shapes.  It's a question of 

\screenshot{irregularboundary}{An irregular decision boundary}

\subsubsection*{Choosing parametmers to fit the data}

\screenshot{supervised-logistic-regression}{Supervised logistic regression}

The \idx{cost function} defined for the supervised learning problem.
In the linear regression case we used the cost function:


\[
     J(\theta) = \frac{1}{m} \sum_{i_1}^m \frac{1}{2} \parens{h_{\theta}(x^{(i)}) - y^{(i)}}^2
\]


We won't use that now.  Instead we'll  define it as:


\[
     J(\theta) = \frac{1}{m} \sum_{i_1}^m \mbox{Cost}\parens{h_{\theta}(x^{(i)}) - y^{(i)}}^2
\]


where 


\[
 \mbox{Cost}\parens{h_\theta (x^{(i)}),  y^{(i)}} = \frac{1}{2} \parens{h_\theta(x^{(i)}) - y^{(i)}}^2
\]


which can be simplified even more:

\[
 \mbox{Cost}\parens{h_\theta (x,  y)} = \frac{1}{2} \parens{h_\theta(x) - y}^2
\]

It's just the square difference.  That worked fine for linear
regressjon, but for logistic regression that isn't so good since it's
a \idx{non-convex} error function.  It has many minima and it's hard
to run a gradient descent function on it since it won't find  a global minimum


\screenshot{nonconvex}{Convex v.s. non-convex cost functions}

We would like a different, convex, cost function so we can use
gradient descent.   One such cost function is:

\[
\mbox{Cost}(h_\theta(x), y)) = \left\{
\begin{array}{rcclcl}
-\log(h_\theta(x)) &\mbox{if} &y&=& 1 \\
-\log(1-h_\theta(x)) &\mbox{if} &y&=& 0 \\
\end{array}\right.
\]

\screenshot{logregcost}{Half the  cost function for logistic regression}

\screenshot{logregcost1}{The other half of the cost function for
  logistic regression}

Convexity analysis is not within the scope of this course, but convex
cost functions are nice.    Next we'll simplify the notation for the
cost function and based on that work out a gradient descent algorithm.


\subsection{Simplified cost function}

The cost function is \[
\mbox{Cost}(h_\theta(x), y)) = \left\{
\begin{array}{rcclcl}
-\log(h_\theta(x)) &\mbox{if} &y&=& 1 \\
-\log(1-h_\theta(x)) &\mbox{if} &y&=& 0 \\
\end{array}\right.
\]

y is either zero or one.  Because of this we can simplify the cost
function.  In particular we can compress the two lines as 

\[
\mbox{Cost}(h_\theta(x), y)) = -y \log(h_\theta(x)) - (1-y) \log(1 - h_\theta(x))
\]

You can prove this by plugging in one for y  in the equation above,
that makes the second part disappear.   Putting zero in will let the
first term go to zero, and that's the same as the function above.

\screenshot{simplifiedcost}{A simplified cost function for logistic regression}

\newcommand{\ivar}[1]{{#1}^{(i)}}
this gives us the actual cost function:
\[
\begin{array}{lcl}
     J(\theta)  &=&
     \frac{1}{m}\sum_{i=1}^m\mbox{Cost}\parens{h_\theta(x^{(i)}),
       y^{(i)}} \\
 &=&  \frac{1}{m}
%\brackets{
\sum_{i=1}^m \log(h_\theta(\ivar{x}) + (1 -\ivar{y})\log\parens{1 - h_\theta(\ivar{x})}
%}
\end{array}
\]


Now why tis particular function? It's derived from \idx{maximum
  likelyhood estimation} which is nice, it's also \idx{convex}.

Given this cost function, what we'll do is:

\[
\mbox{min}_\theta J(\theta)
\]

\screenshot{logisticregcostfunction}{A cust function for logistic
  regression that is actually useful for something :-)}

now all we need to do is how to find the thetas.  We will use gradient
descent.


The standard template is to use a step within the gradient descent
algorithm:

\[
     \theta_j := \theta_j - \alpha\frac{\partial}{\partial\theta_j} J(\theta)
\]


In our case this means that:

\[
     \theta_j := \theta_j - \alpha\sum_{i=1}^m \parens{h_\theta(x^{(i)})x_j^{(i)}}
\]

(simultaneously update for all \(\theta_j\).  This looks identical to
lienar regression!

\screenshot{graddesclogreg}{Gradient descent with logistic regression}

We can use the same techniques for monitoring to make sure that this
regression is progressing nicely.  Slow descent of the error J is what
we are looking for.

we should try to use vectorized implementations if we can.

\screenshot{vecorizedlogisticregression}{A vectorized implementation
  of logistic regression}

feature scaling can help both linear regresson and logistic
regression to run faster.  

Logistic regression is very powerful and perhaps the most used
classification algorithm in the world and now I know how to work with
it myself :-)


\subsection{Advanced optimization}

With these techniques we'll get logistic regression to have better
performance: Better running times, more features.  We have a cost
function, but we need to have code that can produce partial
derivatives

Strictly speaking we don't need the actual J function, but we'll
consider that case anyway since it is very useful for monitoring
progress and convergence.

When we've got code for J and the partials, we can use many algorithms
to compute a minimum:

\begin{itemize}
\item \idx{Gradient descent}
\item \idx{Conjugate gradient}
\item \idx{BFGS}
\item \idx{L-BFGS}
\end{itemize}

The details of these functions are beyond the scope of this voruse.


Don't need to manually pick \(\alpha\).


They have a clever inner loop called a \idx{line search algorithm}
that picks an efficient \(\alpha\) for us.   Often faster than
gradient descent, but more complex.  In essence, they put a little
regulator on the optimization algorithm to keep it within the sweet
spot. Can choose a different learning rate for each iteration.

Ng used these algorithms for a long while (over a decade), but only
recently did he figure out the details of what they do.

These algorithms are so complex that you probably shouldn't write
these algorithms yourself.  Use a library instead.  Fortunately octave
has a very good library implementing some of these algorithms, so just
use those libraries and you'll get a decent result.

\subsubsection{An example}

\screenshot{exampleforadvancedoptimization}{An example function for
  illustrating how the advanced optimization algorithms can be used}

\screenshot{example2}{example2}

\idx{fminunc} is the octave function for ``function minimization
unconstrained''. initialtheta is the initial guess, and options is a
set of options.  the atsign is a pointer in octave syntax.

\screenshot{optimizingfuncparam}{Using the advanced optimization functions}

In logistic regression we use this like this that gives us the cost
function for logistic regression.

\screenshot{logisicregressionfunc}{An implementation of logistic regression}

Using advanced algorithms is a bit more opaque, but for big problems,
they are bettrer for big problems.

\subsection{Logistic regression on multiclass clasification
  one-vs-all}

This is about the \idx{one v.s. all algorithm}.  Assume we have a mail
classification problem, the classes are work,friends, family hobby.
In a medical system we may diagnose patients not ill, cold flu, or
weather sunny, cloudy, rain, snow.

\screenshot{multiclass-datasets}{A datset containing entries of many classes}

One-vs-al classification works like this:   Assume three classes.
Turn it into three binary classification problems. class one v.s. the
rest, class two v.s. the rest and class three v.s. the rest.

\screenshot{one-vs-all-in-action}{Using the ``one v.s. all''
  classification method}

for each classifiser we fit a classifier:

\[
   h_\theta^{(i)}(x) = P(y=i |  x; \theta)\, (i \in \braces{1,2,3}
\]

We estimate ``what is the probability that x is in class i
parameterized by theta''

To make a prediction, we run all the classifiers and select the one
with the highest probability:

\[
 \max_i h_\theta^{(i)}(x)
\]

``Pick the classifier with the most enthusiasm :-)''


\chapter{Regularization}

\section{The problem of overfitting}

\screenshot{overfitting}{Ovefitting -- The nemesis of curve fitters}

Linear regression and logistic regression can run into \idx{overfitting}
which.   One way of ameliorating the problem is \idx{regularization}.
We'll now look at overfitting and later the fix.

If we get a poorer  fit from linear regression, we get a model that is
\idx{underfit} or has \idx{high bias}.   The term bias ``bias'' is
historical, an indicates that the model has  a strong preconception
that the data will be linear, and  despite the data it will still fit
linear data.   We can then add more factors
and for instance try using higher order polynomials to match the
data.  That will reduce the error to the point that will not have any
error. However the curve will be wiggely, and that curve will be
\idx{overfitted}, or have \idx{high variance}.  The term ``high
variance'' is another technical one.   The space of possible
hypothesis is just too large, to variant, and we don't have enough
data to give a good hypothesis.  In the middle there is the \idx{just
  right} case :-)  The problem with overfitting is that it gives bad
predictors, even if it gives good fit for the training set.  It fails
to generalize.


\screenshot{logisticregressionoverfit}{Overfitting a logistic
  regression classifier}

The problem applies both to linear regression and to logistic
regression.

\subsection{Adressing overfitting}

Overfitting can be recognized by tools we'll learn more about later.
However, there are two mi options for adressing the problem.

\begin{enumerate}
\item reduce number of features
\begin{enumerate}
\item Manually select which features to keep.
\item Model selction algorithm (later in the course)
\end{enumerate}

\item Regularization
\begin{enumerate}
\item Keep all the features, but reduce magnitudes/alues of parameters \(\theta_j\)
\item Works well when we have a lot of features, each of which
  contributes a little  bit to predicting \(y\).
\end{enumerate}

\end{enumerate}

\section{Cost function}

Implementing regularization is a good way to learn regularization.

\screenshot{penlaltyforcost}{How adding a penalty for higher order
  terms reduces overfitting}

Add penalty for higher order factors, meaning that they are only
included in the model if they are really worth it.  The idea is to
have smal values for parameters in \(\theta\).   This gives us
smoother and simpler
hypothesis which are less prone to overfitting.

If we have a bunch of features it is difficult to pick the ones that
are relevant.   So what we'll do is to modify the cost function to
penalize all parameters by adding a regularization term:


\[
 J(\theta) = \frac{1}{2m} \brackets{\sum_{i=1}^m \parens{h_\theta(x^{(i)}) -
   y^{(i)}}^2  + \lambda \sum_{i=1}^n \theta_j^2}
\]


\screenshot{regularizedgradientdescent}{Regularized gradient descent:
  Gradient descent with penalty for higher order terms.}

It's convention to not penalize \(\theta_0\).  \(\lambda\) is the
\idx{regularization parameter} it regulates the tradeoff between the
two objectives of fitting the data well and keeping the parameters
smal.

A too high value of \(\lambda\) is equivalent to fitting a straight
horizontal line.  It will result i severe \idx{underfitting} due to a
too strong bias that the function is  straight horizontal line.



\section{Regularized Linear Regression}

We found this error function.


\[
 J(\theta) = \frac{1}{2m}
 \brackets{\sum_{i=1}^m \parens{h_\theta(x^{(i)}) - y^{(i)}}^2} + \lambda \sum_{i=1}^n \theta_j^2
\]


When using this in gradient descent, we need a new error function and
gradient.  The gradient is very similar to the original, but it has an
extra step:



\[
    \theta_j := \theta_j - \alpha
    \brackets{\sum_{i=1}^m \parens{h_\theta(x^{(i)})x_j^{(i)}} - \frac{\lambda}{m}\theta_j}
\]


Weirdly it's must be true that \(1-\alpha\frac{\lambda}{m}\le 1\)

\screenshot{regularizedgradientdescent}{Gradient descent taking
  regularization into account}


By grouping the \(\theta_j\) factors together, we get:

\[
    \theta_j := \theta_j (1 - \alpha \frac{\lambda}{m}
        \sum_{i=1}^m \parens{h_\theta(x^{(i)})x_j^{(i)}}
\]



We also have the normal equation that can be used to find a
minimization.

\screenshot{normalequationforgradientdescentwithregularization}{The
  normal equation for gradient descent with regularization}

\subsubsection{Non invertibility}


\screenshot{noninvertibleregularized}{A noninvertible/singular
  covariance matrix in the normal equation}

The regular inverse will fail if you have a noninvertible matrix, so
the pinv method is necessary.

Fortuntately, regularization takes care of that problem too, so we
won't have a degenerate (singular) matrix.

\section{Regularized Logistic Regression}

We have both advanced optimization method and gradient descent, and
we'll now learn how to used these for regularized logistic regression.


Logistic regression is prone to overfitting.  All we have to do is to
change the cost function.

\screenshot{regularizedlogisticreg}{Logistic regression with normalization}

How do we implement this?  We treat \(\theta_0\) separately, but then
do something very similar to what we did for linear regression.

The update actually is cosmetically identical to the one we use in
linear regression, but it is actually different since the hypothesis
uses the logistic function, not just a linear function of x.

\screenshot{regularizedgraddeslogistic}{Gradient descent for logistic
  regression with regularization}

The term in the square bracket is the new partial derivative.



How to implement this using the advanced optimization.  First we must
define a cost function

\screenshot{advancedLogisticRegCostFunction}{A cost function to be
  used with the advanced optimization algorithms}

If you understand the stuff presented so far, you probably know as
much machine learning as a lot of engineer working in the silicon
valley.  Still more to learn though :-)  Next up is highly nonlinear
classifiers that we can use.

\chapter{Neural networks representation}

Neural networks are old, were out of favor for a few years, but today
it's the state of the art.

Why? If the decision boundary is highly nonlinear the logistic
regression will be very difficult.  For a quadratic features, the
number of features grows quadratically, and that's a lot of
features. It's both computationally expensive and it's prone to
overfitting.  Same thing for even higher orders.   High orders of
parameters is a pretty normal situation in machine learning.


Computers don't see things s we do, they just se a whole lot of
numbers.    Building a car detector for images is hard.  Training the
classifier for cars needs a lot of parameters :-)

\screenshot{whyvisionishard}{Why computer vision is hard, the computer
only sees numbers and has to create a concept of whatever it sees
based on these numbers.}

\screenshot{representingcars}{Representing cars}

Quadratic regression over a a pixel can easily have several tens of
million of features that are applicable for logistic regression.

\section{Neuron and the brain}

They are biologically motivated, just to get some idea of what they
can do.    Origins in algorithsm to mimic the brain.  Widely used in
80s and 90s, popularities diminished in late 90s.  Recent resurgence:
Stae of the art tehniuqe for many applications.   

The does a lot of amazing things.  There is a hypothesis that the
brain uses a single learning algorithm, the \idx{one learning
  algorithm hypothesis}.  Example, if we route visual nerves to the
auditory cortex, the animal will learn how to see again.  The brain
rewires dynamically. \idx{neuro rewiring experiments} indicates that
different parts of the brain can process different types of
information, so perhaps it is reasonable to assume that all the parts
of the brain uses the same learning algorithm, not a raft of
task-specific algorithm.(metin and frost 1989).

\screenshot{brainsensorrepresentations}{The brain is capable of
  mapping a wide range of sensor inputs into stuff it can make sense of}

\subsection{Model representation}



\screenshot{neuroninthebrain}{The structure of a single neuron in the brain}


Brains are full of neurons.   Neurons have a body and number of input
wires \idx{dendrites}.  The output-wire is called \idx{axon}.    At a
simplistic level it gets a bun of input from its input and send an
output to the output axon.

\zcreenshot{neuronsindthebrain2}{Artist's representation of a neuron
  in a brain}{10cm}

The neurons sends signals called \idx{spikes} to dendrites of other
neurons.   This is the process by which all neural computation
happens.  This is also the way that IO is done to muscles.


In an artificial neural network we model a neuron as a logistic unit:

\screenshot{logisticunit}{Modelling a single neuron as a ``logistic
  unit'' (using the logistic function)}

This is a very (perhaps vastly) simplified version of the network.

Sometimes an extra input \(x_0=1\) is drawn, and sometimes not. It's
called the \idx{bias unit}.   The \idx{activation function} is here
the logistic function.   In a neural network calls the \(\theta\)
parameters \idx{weights}.

The diagram above represents a single neuron.  A network is a group of
neurons thrown together.



\screenshot{neuralnet}{A schematic drawing of a multi-layer neural network.}


The first layer is called the \idx{input layer}, The final layer is
called the \idx{output layer}.  The layer(s)in between is called
\idx{hidden layer}(s) (not observable in the training examples).


\screenshot{nnrepresentation}{Representing a neural network as a
  matrix of weights}

In the network we call the unit \(a_i^{(j)}\) the \idx{activation} of
unit \(i\) in layer \(j\).  The matrix \(\Theta^{(j)}\) are all the
weights controlling the function mapping  from layer \(j\) to layer
\(j+1\).

\subsection{A vectorized representation of neural networks}

\xscreenshot{forwardpropagation2}{Propagating values towards the
  output using ``forward propagation''}

New notation :  \(a^{(2)}_1 = g(z^{(2)}_1) \)  refer to items in layer
2 (the hidden layer)  of the network.

We can use this to vectorize the operation of the calculation of
neural network values.  The activation layer \(a^{(1)}\) is defined to
be the x value (from linear algebra convenson).  We add a \(a_0^{(1)}
= 1 \) as an extra bias input.  Similarly we add a \(a_0^{(2)}
= 1 \) to the hidden layer, being a bias unit there.  To determine the
output value we calculate \(z^{(3)}\).

\mXXX{Why no thresholding?}

\subsection{Intution about what NNs are doing}

\screenshot{coverupNN}{A neural network where a couple of layers has
  been covered up}

A covered up neural network (fig \ref{coverupNN}) is essentially
a logistic regression facility.  The features that are fed  into the
network are the hidden units instead of the (externally input)
features.

The cool thing about this is that the mapping from layer one to layer
two has learned how to recognize features on its own from the input.
And it is these learned features that are then applied to the final
logistic output function.   This is a very flexible mapping function
so it can do a lot more than e.g. a polynomial used in a logistic
regression input.

the \idx{architecture of a neural network} describes how the nodes are
connected.

\subsection{Non-linar classification xor/xnor}

\screenshot{xorxnor}{A neural network capable of calculating the 
  logical function exclusive or/ exclusive nor (XOR/XNOR)}

\screenshot{andnet}{A neural network computing the logical ``and'' of
  two parameters.}

Landmark for the sigmoid function: When the x value is 4.6 the y value
is 0.99.

\screenshot{ornet}{A neural network computing the logical ``or''
  function of two parameters}


We can also mae a network for ``not''

\screenshot{negation}{A neural network computing a logical negation}

\screenshot{xnornet}{A neural network computing the ``exclusive nor''
  (xnor) logical function}

In general, negation is implemented by putting a large negative
weight.

\subsection{Neural networks for multiclass classification}

Digit recognition is a multiclass classificaiton problem. It's a
version of the \idx{one for all} classification method used for
logistic regression.    We assign a node in the output layer to each
of the classes we wish to find.  So we actually get four logistic
regression classifisers representing the final step of the setting.


\screenshot{neuralmulticlass}{A neural network for solving a a
  classification into multiple classes}

The way we do this we represent the training set  as a vector with a
single 1 in it (the rest zeros)  representing the classification of
the item.

\section{Cost functions of neural networks}

NN is one of the most powerful learning algorithms we have.  We'll
start with the const function.  We'll focus on classification
problems.

\screenshot{nncostfunction}{The cost function for a neural network.}

We'll look both on binary classification and multi-class
classifictaion.   Multi-class has multiple output nodes, one per class.

We're going to generalize over the cost function we used for logistic
regression.  Recall that that cost function was:

\[
J(\theta) = - \frac{1}{m} \brackets{\sum_{j=1}^n y^{(i)} 
          \log h_\theta(x^{(i)})     +
          (1 - y^{(i)})\log(1 - h_\theta(x^{(i)}})     
+ \frac{\lambda}{2  m}\sum_{j=1}^n \theta^2_j
\]


For our neural network, we have:

\[
J(\Theta) = - \frac{1}{m} \brackets{
     \sum_{j=1}^m \sum_{k=1}^K
        y_k^{(i)} 
          \log h_\Theta(x^{(i)})_k     +
          (1 - y_k^{(i)})\log(1 - h_\Theta(x^{(i)}_k}   
+ \frac{\lambda}{2  m}\sum_{l=1}^{L-1} \sum_{i=1}^{s_l} \sum_{j=1}^{s_{l+1}} (\Theta_{ji}^{(l)})^2
\]

\screenshot{actualNNcostfunction}{A cost function for a neural network
with regularization terms.}

We sum over all the \(k\) output units, so we sum logistic regression
functions, and that's basically the modification. The regularization
term's summation sums over all the \(\Theta\) values, except that we
don't sum up the bias terms.  Basically we are adding the square of
the weights \mXXX{right?}

\subsection{The backpropagation algorithm}


We wish to find some algorithm to  minimize \(J(\Theta)\) so we need
the partial derivatives of the cost function:

\[
  \frac{\partial}{\partial{\Theta^{(l)}_{ij}}} J(\Theta)
\]

and those are what we'll concentrate on now.     One training example
case:

\screenshot{NNgradientcomp}{Computing the gradient of a neural network}

The \idx{backpropagation} is an algorithm that calculates the
gradient.   Intuitively we have  \(\delta_j^{(l)}\) representing the
``error'' of node \(j\) in layer \(l\).  The activation of the
corresponding node (for \(\delta_j^{(l)}\) ) is 
\(a_j^{(l)}\).

\screenshot {NNbackprop}{Computing the gradient of  a neural network
  using the backpropagation algorithm}

Concretely, for a L=4 node in the network seen in fig \ref{NNbackprop} is:

\[
\delta_j^{(l)} = a_j^{(l)} - y_j =  h_\Theta(x)_j - y_j 
\]

(setting \(l=4\))

A vectorized version of the equation above is \(\delta^{(l)} =
a_j^{(l)} - y_j =  h_\Theta(x) - y\).

We then calculate the \(\delta\) for the earlier layers:


\[
\begin{array}{lcl}
  {\delta_j}^{(3)} &=& ({\Theta}^{(3)})^T {\delta_j}^{(4)}  .*  g'({z}^{(3)}) \\
  {\delta_j}^{(2)} &=& ({\Theta}^{(2)})T {\delta_j}^{(3)}  .*  g'({z}^{(2)}) \\
\end{array}
\]

(where \(.*\) is the elementwise multiplication operator).

And in general   \({\delta_j}^{(k)} = ({{\Theta}^{(2)}}^T{\delta_j}^{(k)+1}  .*  g'({z}^{(k)}) \).
 
The function \(g'({z}^{(2)}\) is formally the derivative of the
  activation function evaluated at the input values given by

\[
  {z}^{(3)} =  {a}^{(3)} .* (1 - {a}^{(3)})
\]

There is no \({\delta}^{(1)}\) since that is the input.

\screenshot{NNbackpropcalculation}{Computing the gradient of a neural
  network using the backpropagation algorithm}

The name \idx{backpropagation} comes from the fact that the algorithm
\idx{propagate}s the errors from the output layer to the previous layers.


Finally, through a somewhat complicated mathematical proof it is
possible to prove that:

\[
  \frac{\partial}{\partial{\Theta^{(l)}_{ij}}} J(\Theta) = {a}^{(l)}_j{\delta}^{(l)+1}_i
\]

(ignoring regularization terms).

\subsection{How to implement backpropagation for a large training set}

\begin{itemize}

\item Assume a training set \(\braces{({x}^{(1)}, {y}^{(1)}), \ldots, ({x}^{(1)}, {y}^{(1)})}\).
\item Set \({\Delta}^{(l)}{ij} = 0\) for all \(l,i,j\).  This wil be
  used to compute the partial derivative term   \(\frac{\partial}{\partial{\Theta^{(l)}_{ij}}} J(\Theta)\)
\item Next we'll loop through the training set \(({x}^{(i)},  {y}^{(i)})\)
\item For \(i=1\) to \(m\)
\begin{itemize}
 \item set \({a}^{(1)}={x}^{(1)}\)
 \item compute forward propagation to compute \({a}^{(1)}\) for \(l =   2,3,\ldots,L\)
  \item Using  \({y}^{(i)}\), compute \({\delta}^{(L}={a}^{(L)} -{y}^{(i)}\)
 \item Then compute \({\delta}^{(L)-1}, {\delta}^{(L)-2}, dots^,( ){\delta}{2}, \)
 \item \({\Delta}^{(l)}_{ij} := {\Delta}^{(l)}_{ij} + {a}^{(l)}_j {\delta}^{(l)+1}_j \)
\end{itemize}
\item 
\[
\begin{array}{lclrr}
{D}^{(l)}_{ij}  &:=& \frac{1}{m} {\Delta}^{(l)}_{ij} +\lambda XXX^{(\Theta)}{l}_{ij} &\mbox{if}& j\neq 0\\
{D}^{(l)}_{ij}  &:=& \frac{1}{m} {\Delta}^{(l)}_{ij}    &\mbox{if}& j=0\\
\end{array}
\]
\end{itemize}


We use the \(\Delta\) terms to accumulate the error terms.  The final
step can be vectorized using:

\[
{\Delta}^{(l)} := {\Delta}^{(l)} + {\delta}^{(l)+1} ({a}^{(l)})^T
\]

While the formal proof is quite complicated, it is in fact true that 

\[
\frac{\partial}{\partial{\Theta^{(l)}_{ij}}} J(\Theta) = {D}^{(l)}_{ij}
\]


This means that we an use these facts in both gradient descent and the
more advanced optimization algorithms.

\subsection{The mehanical steps of backpropagation}

Backpropagation is a less mathematically clean algorithm, so it's ok
to be a  bit confused :-)

\screenshot{NNForwardIntuition}{NNForwardIntuition}

Backpropagation is a lot like running forward propagation backwards.

\subsection{Implementation note: Unrolling}

\screenshot{NNunroll}{The weight matrices are rolled into / unrolled
  from rows in matrices }


The \idx{fminunc} like algorithms of this world requires vectors as
inputs and deliver them as outputs.   But in NNs we are working on
matrices.


\cleardoublepage

\screenshot{NNunroll1}{With the unrolled thetas, we can use standard
  advanced optimization techniques to find parameters for the neural network}


The fix is to unroll them into vectors.  Remember that the indexes
that are used when rolling/unrolling are {\em inclusive}, so as always
beware of off-by-one errors caused by bogus intuition (it's an easy
thing to do).

\screenshot{NNUnrolledLearningAlgorithm}{The learning algorithm using
  an unrolled representation}

\subsection{Gradient checking}

There are many ways to get subtle errors in backpropagation.   It's
also not so easy to know about it.  There is a trick called
\idx{gradient checking} that eliminates many of these problems and
thus gives you a higher confidence that your implementation don't have
bogus bugs :-)


\screenshot{NNgradientChecking}{Gradient checking of the NN
  algorithms, just to make certain that things are on track.}

Compute \(\theta +/ \epsilon\) and connect them through a straight
line and use the slope as an approximation of the derivative. There is
a range of epsilons that will work, but you don't want it to be {\em
  too} small since that will lead into numerical problems.

The two-sided estimate (+/) gives better estimate than the one-sided
one.  This gives us a numerical estimate of the gradient at some
point.

We can then estimate all the partial derivative terms for a whooping
big vector \(\theta\) as shown in fig \ref{NNGradUnrolled}.  The trick
is to calculate the partial change per variable, and use that to
produce the partial derivative for the whole parameter set.

\screenshot{NNOctaveGradientCheck}{Even more gradient checking in Octave}

Take the gradient we get from backprop and check if it's more or less
similar to the gradient approximation.

Prof Ng's standard way of implementing BP is:

\begin{itemize}
 \item {\bf Implementation note:}
\begin{itemize}
\item Implement backprop to compute DVec (unrolled \(D^{(1)}, \ldots,  D^{(1)} \)).
\item Implement numerical gradient check to compute gradApprox.
\item Make sure that they give similar values.
\item Turn off gradient checking.  Using backprop code for learning
\end{itemize}

 \item {\bf Important::}
\begin{itemize}

\item Be sure to disable your gradient checking code before training
  your classifier.  If you run numerical gradient computation on
  every iteration of gradient descent (or in the inner loop of
  ``costFunction'', your code wil be {\em very} slow.

\end{itemize}


\end{itemize}


Backpropagation is actually a very computationally efficient algorithm
to calculate the gradient.


\subsection{Random initialization}

\screenshot{NNinitialtheta}{Initializing the thetas using random numbers}

There is one final idea we need to get this thing going, and that is
\idx{random initialization}.  We need some initial theta.  Zero is
actually not an option for NN (even if it worked for logistic
gregression) since it  all deltas will be zero, and we won't get a
gradient to traverse.   All the hidden units computes the same values
initially and that's not working.  The delta values will all be the
same.  The partial derivatives will be equal.  That means that the
weights will be the same even if they are non-zero.  In essence, we
won't get efficient learning. Not only will the weights be identical,
but all the outputs from all the hidden units will compute the exact
same feature.  This is a highly redundant representation and it
prevents the network from  learning something interesting.

This is called the \idx{problem of symmetric weights}.

We get around the problem by setting the initial weights to be some
small random number (see fig \ref{NNrandominitialized} for an octave
we perform \idx{symmetry breaking}implementation), and this is how we
perform \idx{symmetry breaking}.

The \(\epsilon\) we have here is different from the \(\epsilon\) used
during gradient checking, hence the name \verb!INIT_EPSILON!.

\subsection{Putting it together}

\subsubsection{The architecture}
\screenshot{NNarchitecture}{The ``architecture'' of a neural network
  is the number of layers, and the way in which the layers are connectec}

Pick a network architecture. The number of features should be the
number of output nodes.   The number of input units should be the
number of features.  As a default, just have one hidden layer.  If you
have more the same number of hidden units in every layer.  Usually the
more the better but the computational cost can be prohibitive.

A bit more nodes than the number of input features is usually ok for
hidden layers.

There will be more about how to do this later.

\subsubsection{Training a neural network}



\screenshot{NNTrainingForloop}{NNTrainingForloop}

\begin{enumerate}
\item Randomly initialize weights.
\item Implement forward propagation to get \(h_\Theta(x^{(i)})\) for
  any \(x^{(i)}\).
\item Implement code to compute the cost function \(J(\Theta)\).
\item Implement backprop to compute partial derivatives
  \(\frac{\partial}{\partial \Theta^{l}}_{jk} J(\Theta)\). Use a
  forloop like the one in figure \ref{NNTrainingForloop}.
\item Use gradient checking to compare \(\frac{\partial}{\partial
    \Theta^{l}}_{jk} J(\Theta)\). computed using backpropagation vs
  using numerical estimate of gradient \(J(\Theta)\).  Then disable
  the gradient checking.
\item Use gradient descent or an advanced optimization method with
  backpropagation to try to minimize \(J(\Theta)\) as a function of
  parameters \(\Theta\).
\end{enumerate}


Even if the functions can't actually guarantee to find a global
optimum, but in fact the results are usually quite well even when the
error function is non-convex.

\screenshot{errorsurface}{errorsurface}

The cost function measures how well the network matches the training
data.   There are parameter settings that approximate well and others
that don't.   

Neural networks is a complex algorithm.   It's quite normal to not
have a full grasp of what the algorithm does.  It's hard to get a
handle on it compared with for instance linear regression.  However,
backpropagation is a very powerfl learning method that is able to fit
nonlinear data to your classifier and this is one of the most powerful
learning algorithms we have today.

\subsection{Autonomous Driving}


\screenshot{autonomousdriving}{autonomousdriving}

Dean Pomerleau  at \idx{Carnige Mellon University} made a self-driving
car based on neural networks. In figure \ref{autonomousdriving} the
topmost input on the left is the steering input from the human, the
lower bar is the output from the neural network.  lower left window
shows the video input that the camera sees.  Before the network starts
learning it outputs a gray fuzz corresponding to the random
initialization.

Only after it has learned for a while it starts steering in  a
distinct direction.

%XXX 
\xscreenshot{alvin}{alvin}


Takes a picture every second.  Reduced to 30 times 30 image (900
pixels).  A three layer netwrk.  Initial steering response is random,
but after about two minutes the network accurately replicates the
steering reaction of the human driver.

When running it uses twelve images per second.  There is a ``run''
switch that must be pushed.  The network generates both a steering
direction and a confidence value.   There are multiple networks
running at the same time, and the most reliable is chosen.   At an
intersection   As the vehicle approaches the intersection the
confidane of the one-lane netowkr decreases.   As it sees a two-lane
road the two-lane network is selected to steer to sfey guide the
network into its lane.

\screenshot{alvinsensors}{alvinsensors}
\screenshot{alvinontheroad}{alvinontheroad}
\screenshot{singlelaneroad}{singlelaneroad}

\xscreenshot{alvinfromtheinside}{alvinfromtheinside}
\xscreenshot{alvinfromtheinside2}{alvinfromtheinside2}{alvinfromtheinside2}

There are more modern drivers than this, but it's still pretty amazing
that a simple neural network can be trained to drive a car somewhat well.

\chapter{Advice for applying machine learning}

How to apply the right algorithms. How to avoid wasting time on
non-promising avenues.  This section will give a bunch of practical
guidelines.   

We'll consider minimizing regularized linear regression:

\[
 J(\theta) = \frac{1}{2m} \brackets{\sum_{i=1}^m \parens{h_\theta(x^{(i)}) -
   y^{(i)}}^2  + \lambda \sum_{i=1}^n \theta_j^2}
\]


What to do if we find unacceptably large errors in the predictions?
What should we do?

Some possible fixes are:


\begin{itemize}
\item Get more training data.  Often it helps, sometimes it doesn't.
\item Try a smaller set of features. If we have overfitting that is
  something we can do.
\item Get more features.  Could work, but it would be nice to know in
  advance if it would work.
\item We could add polynomial features.
\item We could increase \(\lambda\).
\item We could decrease \(\lambda\).
\end{itemize}

Unfortunately the most used method for selecting among these is gut
feeling.  Unfortunately this is not very efficient.  Fortunately there
is a fairly simple technique that can easily rule out about half of
these options.   We'll learn about these \idx{machine learning
  diagnostics}.  Diagnostics can take time to implement, but doing so
can be a very good investment since it helps making choices when
improving the algorithm and many months can be saved that way.

\section{Evaluating an hypothesis}

When fitting data to the hypothesis we are trying to minimize the
training error.   Too many parameters in a polynomial will not be very
useful since we are overfitting.  For few parmeters we can use
inspection, but for high number of parameters that isn't practical.

A more general method is to split the data set into a training set and
a test set. A typical split is 70/30.  The number \(m_{\mbox{test}}\)
is the number of items in the test set. The training set is used to
train the algorithm (learn the parameter vector)., and the test set is
used to evaluate its performance.  If there is any kind of ordering in
the data, it is better to split the data randomly rather than use the
sequence they arrived in.

\begin{itemize}
\item Learn the parameter \(J(\theta)\) using the training data
  minimizing the training error.  You use 70\% of the test learning
  data for this.
\item Then you compute the test set error: \[
   J_{\mbox{test}}(\theta) = \frac{1}{2 m_{\mbox{test}}}
   \sum_{i=1}^{\mbox{test}} \parens{h_\theta(x^{(i)}_{\mbox{test}})- y x^{(i)}_{\mbox{test}}}^2
\]

Use the squared sum of errors or something similar.
\end{itemize}

How about if we were using a classification problem? It's very
similar:

\begin{itemize}

\item First we learn the parameter \(\theta\) from the training data.
\item Then we compute the test set error: \[
   J_{\mbox{test}}(\theta) = - \frac{1}{m_{\mbox{test}}}
   \sum_{i=1}^{m_{\mbox{test}}}
   \log h_\theta(m_{\mbox{test}}) + 
   (1 - y_{\mbox{test}}^{(i)} \log h_\theta(x_{\mbox{test}}^{(i)})
\]

The same objective function that we always use for logistic
regression, but use \(m_{\mbox{test}}\). 

Most of the time this is a perfectly good way of calculating
classification errors, but sometimes there is another metric that is
also useful:

\item Misclassification error (0/1 misclassification error):  0/1
  denotes that an example is either right or wrong:

\[
\mbox{err}(h_\theta(x),y) =
 \left\{
\begin{array}{lll}
1  & \mbox{if}& h_\theta(x) \geq 0.5, y = 0, \mbox{or} h_\theta(x) \le 0.5 \\
0 & \mbox{otherwise}&\\
\end{array}
\right.
\]

Or in other words.  If the prediction was wrong the function gives
``1'' and if it was right it gives ``0'', so summing all the errors
will give the number of actual errors.   We can then let the tes
error be defined as:  

\begin{verbatim}
\[
\frac{1}{m_\mbox{test}} \sum_{i=1}^{m_\mbox{test}}
\mbox{err}(h_\theta(x_\mbox{test}^{(i)}), y^{(i)})
\]
\end{verbatim}

The error is the fraction of the misclassifications.

\end{itemize}


That's the standard technique for testing how good a learned
hypothesis is.

\section{Model selection and training/validation/test sets}


How to select the regularization parameter, number of terms in a
polynomial etc.?  These questions are called \idx{model selection
  problems}. We'll learn how to split the data into \idx{training set}
\idx{validation set}  and \idx{test set} and use these to do model
selection.


We have already seen  many examples of overfitting.   This is why
training set errors are not a good predictor for how the data will fit
other data and generalize to other data not seen in the training set.

In general, the error on any error data is unlikely to be a good
predictor of generalized error.

\screenshot{tendegpoly}{A ten degree polynomial can fit many a crooked
  curve, but should it?}

Consider the case where we are looking at polynomials with up to tenth
degree terms.  The model selection problem we are trying to solve is
to figure out which degree of polynomial, denoted \(d\) we should use
to model the problem.

Here is one thing we can do.  First take the first model and minimize
the training error, and do the same thing for increasing values of
\(d\).  We can then take the parameters and compute the test set errors.
We can then see which model has the lowest tett set error.      The
problem is that this will probably not be a fair comparison since we
fit the parameter for giving the best possible performance on the test
set, so the performance on the of the parmeter is likely to be overly
positive, so we need another test set, called the \idx{cross
  validation set} (CV), sometimes called \idx{validation set}.   A
typical ratio would be 60/20/20. These numbers can vary but this is
typical.

\screenshot{tvterrors}{tvterrors}

When faced with a model selection problem, we'll use the cross
validation set to select the model.  We'll select the model with the 
{\em lowest} cross validation error.  Then finally we'll use the  \idx{test
set} to determine the performance of the estimator.



In machine learning as it is practiced the model is often selected
using the test set and reporting the error on the testset.   Most
practitioners advise against this.  If you don't have a really huge
testset this is most likely a pretty bad idea.

\section{Diagnosing bias v.s. variance}

If you run a learning algorithm and it is not doing as well as hope,
it is almost always because it has a \idx{high bias problem} or a
\idx{high bias problem}. In other words, either an underfitting
problem or an overfitting problem.  It is very important to figure out
which (or both) of these you have to improve your algorithm.

\screenshot{biasvariance}{biasvariance}

\screenshot{biasvarianceplot}{biasvarianceplot}

We'll now consider the error as a function of the model selection
parameters.  The plot can be found in figure \ref{biasvarianceplot}.

Increasing the degree of the polynomial wil typically let the training
error fall with increasing degree of the polynomial.  However, that
may not be the case for the crossvalidation error. That may in fact
increase with higher model parameters.

\screenshot{biasvariance2}{biasvariance2}

In figure \ref{biasvariance2} the left side corresponds to a low bias
case, the right end corresponds to a high variance problem.  The high
bias case (underfitting) case will give high both training and
validation errors.   If the training error is low and the cross
validation error is low then there is a variance problem (overfit).

The key that distinguishes these two cases the training set error
will be high in high bias problems and it will be low in high variance
problems.

There are in fact four distinct failure modes:

\begin{itemize}

\item High bias (overfitting)
\item High bias (underfitting)
\item High variance (overfitting)
\item High variance (underfitting)
\end{itemize}

It is also possible to suffer from both high variance and high bias
\mXXX{Obviously I don't grok this yet}.

\section{Regularization and bias variance}

Regularization can help  against overfitting, but how does it help
against bias invariance.  Assume we have a high order polynomial with
regularization.

\screenshot{linregularization}{linregularization}

With a high \(\lambda\) we'll get high bias and overfitting.    With a
zero value we get high variance.   


We can use this to automatically chose the regularization value.
Ng usually selects values of \(\lambda\) in steps of .02 starting at
0.01 up to 10.24 (for twelve models).  The same procedure can be used
as for the \(d\) parameters and test them using the cross validation
set for validation.

One typical to test for would be

\begin{itemize}
\item Model: \(h_\theta(x) = \theta_0 + \theta_1x + \theta_2x^2 +
  \theta_3x^3 + \theta_4x^4 \)
\item Penalty:  \[
  J(\theta) = \frac{1}{2m}
  \sum_{i=1}^m \parens{h_\theta(x^{(i)}) - y^{(i)}}^2    +
  \frac{\lambda}{2m}\sum_{i=1}^m \theta_j^2
\]
\item Range to test over \(\lambda \in \braces{0,0.001, 0.02, 0.04,
    0.08,  \ldots, 10.24}\).  (a sequence doubling the \(\lambda\) for
  each step).

\item Calculate
\begin{verbatim}
 \[
\mbox{min}_{\theta} J(\theta) \rightarrow
  \theta^{(i)} \rightarrow J_\mbox{cv}(\theta^{(i)})
\]   
\end{verbatim}
 for all the
  \(\lambda_i\) values.
\item Select the \(\lambda\) that gives the lowest crossvalidation error.
\item Finally calculate the test error for the selected \(\theta^{(j)}\).
\end{itemize}

When choosing the regularization parameter, we use a training error,
cross validation and test error defined {\em without } the
regularization factor (half average squared error, typically).

\screenshot{regularizationbiasvar}{regularizationbiasvar}

It often helps to plot a figure like the one in fig \ref{regularizationbiasvar}.

Next we'll look at \idx{learning curves} to diagnose problems with
learning algorithms.

\section{Learning curves}


\xscreenshot{lcurves}{lcurves}

Learning curves is a very useful tool to gauge the performance of a
learning algorithm

A learning curve is a plot of the training and cross validation errors
as a function of the training set.  The trick is to deliberately
reduce the size of the training set.

\screenshot{quadlearningcurve}{quadlearningcurve}

With small training sets the training errors will be very small.
With larger ms there will be larger and larger errors for the training
set.

For small training sets the cross validation error will tend to
decrease the larger the training set is.


\screenshot{highbiaslearningcurve}{A learning curve that's typical in
  a high bias situation}

If your model can't match the data, the cross validation error will
drop a bit, but then it will flatten and will stay high.   For high
bias, the crossvalidation error and the training errors are high, and
relatively similar.   Consequently, if a learning algorithm has high
bias, adding more data to the training set {\em will  not improve}
performance.  This is a useful thing to know before setting off to
getting tons of extra datan (in this case it may be useful).

\screenshot{highvariancelearningcurve}{A learning curve that's typical
in a high variance situation}

If we are fitting a high order polynomial and a small \(\lambda\) we
will have little training error but as the training set increases the
training set error will still be pretty low.     However, the cross
validation error will be high.  The indiciative diagnostic is that
there is a high gap between the training error and the cross
validation error.  However, adding more data will decrease the gap, so
with high variance learning algorithms adding more data will indeed
help performance.


Plotting learning curves can be useful to figure out if the learning
algorithm is suffering from bias or variance or both.  Plotting
learning curves is always a good idea.

\section{Deciding what to  next revisited}

\begin{center}
\begin{tabular}{lll}
{\bf Cure} & {\bf Fixes problem type} & {\bf Indication}\\ \hline
Get more training examples & High variance  & CV error much larger than training set error.\\ \hline
Try smaller set of features &High variance  &  As above. \\ \hline
Try getting additional features & High bias & ???\\ \hline
Try adding polynomial features (\(x_1^2,(x_2^2, x_1x_2,\), etc. & Highbias&\\ \hline
Try decreasing \(\lambda\)& High bias & Easy to do, just try it\\ \hline
Try increasing \(\lambda\)& High variance & As above  \\ \hline
\end{tabular}
\end{center}



\screenshot{NNOverfittingprotection}{How to protect a neural network
  against overfitting: Add more hidden nodes and use regularization :-)}


Neural networks are also subject to over and underfitting.  Small
networks are more prone to underfitting, but are computationally
cheaper.   Large networks are more prone to overfitting but are also
more computationally expensive.   Use regularization (\(\lambda\)) to
address overfitting.

Using a larger network with regularization is usually the best
option.   

A single hidden layer is usually a good choice, but if you want to try
other architectures, using the cross validation method to perform this
type of model selection is also an option.


\chapter{Machine learning system design}

This main issues facing us when we are designing a machine learning
system.   These videos may seem a bit disjointed, and less
mathematical.    However these issues will most likely be huge
timesavers for real operations.

\section{Prioritizing what to work on}

\screenshot{spamnospam}{An example of Spam, and an example of non-spam
(a.k.a. ``ham''}

Assume that we are working on a spam classifier.    We classify the
examples. How do we apply supervised learning?     The first thing we
must decide is how to represent the features.  

One way is to choose 100 words that are indicateive of spam/not spam
(e.g. ``deal'', ``buy'', ``discount'' etc.).   Whereas email
containing a correctly spelled first name may be indicative of
non-spam.

Given a piece of email we can then encode it as a feature vector.  We
can then define a feature to be the word count for the words in our
vocabulary. We can even let the word-count be binary, ``0'' if the
word doesn't appear, and ``1'' if it appears, however many times the
word appears.  This means that \(x_j\) says if the word \(j\) is
present or not.  Typically the number of word is much higher, say  ten
to fifty thousand automatically picked for high frequency.

Question: How should we spend our time?

\begin{itemize}
\item Collect lots of data (e.g. ``honeypot'' project).  This will
  only be useful sometimes.
\item Develop sophisticated fetures basd on email routing information
  (from the email header).  Often spammers try to obscure the origin,
use unusual routes etc.  The header can be
used to develop signals.
\item Develop sophisticated features for message body, should
  ``discount'' and ``discounts'' be treated as the same word? How
  about ``deal'' and ``dealer''? Features about punuctuation?
\item 
\end{itemize}

\screenshot{quizzzz}{How to select the best way to solve a machine
  learning problem.}

Very often one can brainstorm a lot of interesting things to try to
use.  However, in general it is not easy to say which possible
predictor works the best.  Listing up the options think about them is
a good idea.  The most commonly used method is to just get up one
morning and then just randomly fixate on some method one day and then
spend six months pursuing that method is not such a great idea.

Techniques of error analysis to select the options to pursue  will
help us guide the method selection :-)

\section{Error analysis}

The recommended approach:

\begin{itemize}

\item Start with a simple algorithm that you can implement quickly.
  Implement it and test it on your cross-validation data.  At most one
  day (24 hours).   Quick and dirty. Test it on the cross validation data.

\item Plot learning curves to decide if more data, more features etc
  are likely to help.    Figure out if the algorithm is suffering from
  high bias, high variance and so on.   

  In general it is very difficult to in advance, in the absence of any
  evidence, to see where time should be spent.  It is often by a very
  simple implementation and interpreting this that these decision can
  be made.  Think of this as avoiding \idx{premature optimization}.

  We should let evidence rather than gut feeling guide where we should
  spend our effort.


\item Error analysis: Manually examine the examples (in the cross
  validation set) that your algorithm made errors on.  See if you spot
  any systematic trend in what type of examples it is making errors
  on.

  This process will often inspire us to design new features or improvements.

\end{itemize}


As an example of error analysis. Ng is quoting an example where you
have 500 crossvalidation samples.  The algorithm misclassifies 100
emails.  \idx{Manually examine} each of the errors and categorize them
based on what type of mail it is, and what cues (features) you think
would have helped the algorithm to classify them correctly.   

Example: Pharma, replica, steal passwords (phishing).   Count them
up, and classify for all of these classes.   Then concentrate on the
most important stuff (e.g. the ones with password stealing).  Some
possible features are deliberate misspeppling (m0rgage, med1cine),
unusual routing, unusual punctuation etc.  Look for strong signals,
and concentrate on those.  ``hillclimbing'' through a manual process.

What we  want to figure out is what is the most difficult to
classify and concentrate onthose.


The importance of numerical evaluation:  Should
discount/discounts/discountes be treated as the same word?    We can
use \idx{stemming} sofware, e.g. the \idx{Porter stemmer}.   But will
it help?   It can hurt since there are false positives (universe =
university).  Error analysis may not be h elpful to deciding.  The
only solution is to try it to see if it works.

In order to do this test it is very imprtant to have a numerical way,
so we can run with and without stemming (\idx{A/B testing}).   For
this example the single row number cross validation is a good test,
but in other cases it is not and it is more difficult to figure out
what to do.

Another case, should we distinguish between upper and lower case?   Do
the test, do the math, decide rationally.   If we need to do manual
evaluation of everything it is harder to try out new ideas than if we
can automatically do error analysis on the cross validation set.

To wrap up: Quick and dirty. Usually people spend too much time on
this implementation.  Don't worry about being too dirty or too quick.
Then use the tool to decide where to spend the time on further
development.  If we have a single row number evaluation metric it's
easy to evolve the algorithm.


\section{Error metrics for skewed classes}

\screenshot{cancerpredictor}{cancer predictor}

We train a logistic regression model to 1 percent error, 99 percent
correct diagnosis. However, only 0.5 percent of the patients really
have cancer.  This means that a predictor that always predicts no
cancer will only have .5 percent error, and that makes our one percent
predictor looking not so hot.

This is the case of \idx{skewed classes} where one of the classes is
much more common than the other.  The problem with using
classification accuracy is that it's hard to use just classification
accuracy to determine the efficiency of the classifier.

We could use another metric is the \idx{precision/recall}
classifier. where y=1  in presence of a rare class that we want to
detect.    The actual class is going to zero or one. Our learning
algorithm will predit som e value for the class, and our learning
algorithm will predict either one or zero.   We get both true and
false positives and negatives.  one way of prediction the performance
is to calculate precision and recall:

\screenshot{precisionrecall}{Precision/Recall}

\begin{itemize}
 \item Precision.  The number of true positives divided by the number
   of predicted positives (true positives + false positives).

\item Recall.   The fraction of patients that actually have cancer,
  what fraction did we correctly detect as having cancer?  True
  positives/(actual positives).
\end{itemize}

More generally it is not possible for an algorithm to cheat by doing
some simple thing by setting values to a constant.  We are much more
certain that a classifier with high precision and recall is a good a
classifier.



\section{Trading off precision and recall}
\zcreenshot{precisionrecalltradeoff}{A tradoff between precision and recall}{70mm}

In some cases we wish to trade off between precision and recall and
here we will learn a bit more about how to do that.  First a bunch of
definitions:

\begin{itemize}
\item Logistic regression \(0 \leq h_\theta(x) \leq 1\).
\item Predict 1 if\(h_\theta(x) \geq 0.5 \)
\item Predict 0 if\(h_\theta(x) \le 0.5 \)
\item Precision:  
\[
\frac{\mbox{True positives}}{\mbox{No of predicted positive}}
\]

\item Recall
\[
\frac{\mbox{True positives}}{\mbox{No of actual positive}}
\]
\end{itemize}


We'll continue with the cancer classification (1 = cancer).  What if
we want to predict y=1 (cancer) only if we are very confident.   One
way to do this is to modify the algorithm by setting the threshold
value at .7 not .5.   If we do this then we will predict cancer only
when we are more confident, so precision will go up. On the other
hand, there may now be more cases that are actually cancer but we
don't recognize as such, so recall may go down.  We can even set the
threshold to be .9, which means that we'll get eve higher precision,
but even less recall.



\begin{figure}
\begin{tabular}{l|lll|l}
                     & Precision(P) & Recall(R) & Average & \(F_1\) score \\ \hline
Algorithm 1  &       0.5        &       0.4    &    0.45   & 0.444 \\ 
Algorithm 2  &       0.7        &       0.1    &    0.4     & 0.175 \\ 
Algorithm 3  &       0.02      &       1.0    &    0.51   & 0.0392 \\ 
\end{tabular}
\label{prscore}
\caption{Various ways of comparing precision/recall}
\end{figure}



Suppose instead that we wish to make really certain that we get as
many as possible patients with cancer. so we want to try hard to avoid
false negatives, then we can do that too by tweaking the threshold,
this time by decreasing it.  Perhaps .3 (30 percent).

Higher precision, increase threshold, but in general the
precision/recall curve looks different for each classifier.

A question is. Can we choose the precision/recall parameters
automatically?  The \(F_1\)  score (\(F\) score) helps us doing that.
It is really useful to have a single real number to indicate how good
our classifier is, but with precision/recall we have actually lost
that and instead gotten two numbers.  One thing to try is the average
of precision and recall, but that is in fact not a very good option
since it gives classifiers that just give out constants really good
scores (illustrated by algorithm 3 in figure \ref{prscore}).  Instead,
there is a formula called ``\(F_1\)  score'' that is much more useful.

\[F_1 \mbox{score} = \frac{P\cdot R}{P + R}\]

It takes the product, so if both precision and recall are zero, the
result is zero.  If both precision and recall is one (perfect), then
the F score would be one.


There are many possible ways of combining precision
and recall, but traditionally the F-score has been used in machine
learning. Also the name doesn't mean anything (the F has no function,
so to speak).  

Also, it's a good idea to do the scoring on the cross-validation data,
not on the training set.  Testing various thresholds based on f-scores
on the cross-validation set would also be a reasonable way to selecdt
a classifier.

\section{Data for machine learning}

\screenshot{bankobrill}{Banko an Brill's result: There is no data like
  more data.  More data will often turn a mediocre machine learning
  algorithm into a decent one.}

How much data to train on?   Ng cautions against always collecting a
lot of data in every case.  In some cases it is not necessary, but on
some cases it is not. In this section we'll discuss this some more.

Banko and Brill in 2001  evaluated different algorithms (perceptron (logistic
reression), winnow, memory-based and naive Bayes)  to separate
between confusable words in english sentences.  The exact details of
the algorithms are not that important for this example.  They then
tried the algorithm on various training set sizes, and the results can
be viewed in figure \ref{bankobrill}.  The trend is very clear.
Initially all of the algorithms give really similar performance.
Also the performance monotonically increase with larger training set.
It's also true that an otherwise inferior algorithm will give very
good results given a larger training set.   This finding has given
cause to a saying in machine learning:

\begin{quote}
\em  It's not who has the best algorithm that wins. It's who has the
most data.
\end{quote}

But when is this true and when is it not true?   

\subsection{A rationale for large data}

Assume a feature "x'' in an n+1 dimensional real space, and that gives
enough information to predict an ``y'' accurately.  Example: ``For
brekfast I ate XXX eggs''.  Counterexample: Predict housing price from
only size in square feet and no other features.

Useful test: Given the input x can a human expert confidently predict
y?  If no, then more data won't help much.

Use a learning algorithm with many parameters (e.g. logistic
regression / linear regression with many features or neural network
with many hidden units).  This will be  algorithms with a low bias.
If we train this algorithm on a large sample, so the training error
will be small.   If the training set is much larger than the number of
parameter we are likely to not overfit, so the training set will
hopefully be similar to the test error.   

We are addressing the bias problem with a low-bias algorithm, and by
using a large learning set we get a low variance (hopefully).

\begin{itemize}
\item Can a human look at the data and confidently predict things?
\item Can we get a large learning set?
\end{itemize}

Can we do both this will often give us a high performing algorithm.

\chapter{Support vector machines}

\section{Optimization objective}

SVM sometimes gives  a cleaner way of learning nonlinear functions
than the methods we have seen so far.  This is the last algorithm for
supervised learning we will spend significant time on.  The support
vector machine is in its essence a modified logistic regression
method.  

Recall that \( h_\theta = \frac{1}{1 + e^{-\theta^T x}} \), for
simplicity we will let \(h_\theta = g(z)\) where \(z = \theta^T x\).
If \(y=1\) we want  \(h_x \approx 1,\) and \(\theta^T x \gg 0\).  The
reason is the output of logistic regression becomes close to one.
Conversely we wish If \(y=0\) we want  \(h_x \approx 1,\) and
\(\theta^T x \ll 0\).   The cost function for logistic regression is
designed so that each pair \((x,y)\) in the training set contributes
(ignoring 1/m).

\[
 -(y \log h_\theta(x) + (1 - y ) \log (1 - h_\theta(x))) = 
  -y \log\frac{1}{1 + e^{-\theta^T x}} - 
  (1 - y) \log (1 - \frac{1}{1 + e^{\theta^T x}})
\]


\xscreenshot{svnerrorfunction}{The error function for a support vector machine}

Now, if y=1 only the first term (on either side of the equality)
counts.  That means that the function  shown in \ref{svnerrorfunction}
is what drives (one) part of the error.  We will now modify that to
create the support vector machine basis .-)  The modification is to
make a pickwise linear function (drawn in purple in the figure) that
is zero from z values of 0 and upwards, and follows a linear
approximation if the (component) of the error function for logistic
regression downwards to infinity.  The slope of the straight line is
btw not very important.  The fact that it's linear gives us increased
computation efficiency.


\xscreenshot{svmerrorfunction2}{The error function for a support
  vector machine}

For the other part of the error function  we use another function that
is basically the mirror of the first.  We call these functions
``\(\mbox{cost}_1(z)\)'' and ``\(\mbox{cost}_0(z)\)''.  Armed with
this we can now build the support vector machine.

The error function for logistic regression is:

\[
  \mbox{min}_\theta \frac{1}{m}\brackets{
 -(y^{(i)} \log h_\theta(x^{(i)}) + (1 - y^{(i)} ) \log (1 - h_\theta(x^{(i)})))}
+ \frac{\lambda}{2m} \sum_{j=1}^n \theta_j^2
\]

For the support vector machine it is:


\[
  \mbox{min}_\theta \frac{1}{m}\brackets{
 -(y^{(i)}\mbox{cost}_1(\theta^Tx^{(i)})) + (1 - y^{(i)} ) \mbox{cost}_0(\theta^Tx^{(i)})}
+ \frac{1}{2m} \sum_{j=1}^n \theta_j^2
\]

And this defines our minimization problem.   By convention the formula
above is reparameterized a bit when used in SVMs, the \(1/m\) term is
removed, since it is just a constant. Furthermore for logistic
regression the objective function is parameterized as \(A + \lambda
B\) by modifying \(\lambda\) we can trade the number of parameters
with precision.  For SVMs we  will instead optimize a function
parameterized by \(C A + B\). This is just a different way of
controlling the tradeoff.

\[
  C \mbox{min}_\theta \brackets{
 -(y^{(i)}\mbox{cost}_1(\theta^Tx^{(i)})) + (1 - y^{(i)} ) \mbox{cost}_0(\theta^Tx^{(i)})}
+ \frac{1}{2} \sum_{j=1}^n \theta_j^2
\]

That gives us our overall cost function for a support vector machine.
In a sense one can say that \(C= 1/\lambda\), or at least the role it
plays is similar, using either one will not change the optimal value
that we will find.

Finally, unlike logistic regression the SVM doesn't output a
probability.  Instead it just makes a prediction:

\[
 h_\theta (x) = \left\{
\begin{array}{lcl}
1&\mbox{if}& \theta^Tx \geq 9\\
0&\mbox{otherwise}&\\
\end{array}
\right.
\]

\section{Large margin intuition}

\screenshot{svmdecisionboundary}{svmdecisionboundary}

SVNs are sometimes called \idx{large margin classifiers}.  A logistic
classifier switches its prediction at zero (when \(\theta^T x \geq 0\)
becomes or stops being true).  An SVM's cost function components on
the other hand requires \(\theta^T x \geq 1\) to predict and
\(\theta^T x \leq -1\).

This builds an extra safety margin into the SVN machine.   Let's
consider a case where the constant \(C\) is very large, the minimation
objective will be highly motivated to find a value where the cost
functions (not the regularization terms) is very close to zero.  Now,
what will it take to make that happen? 


\screenshot{linsepSVM}{A support vector machine with a linear kernel}

For an SVN the decision boundary will be linearly separably.  The
decision boundary for the SVN has a large \idx{margin} between the
classes (drawn in blue in \ref{linsepSVM}).  The optimization problem
we've just seen actually creates this \idx{large margin classifier}.
The SVN classifier will try to separate the positive and negative
examples with as big a margin as possible.



\screenshot{outliersensitivitySVM}{SVM and outliers, how sensitive
  should one be to them?}

SVM is actually more sophisticated than it seems.  If the C is very
large, then the classifier will be very sensitive to outliers, but if
a more reasonable C is used, then it can be made less sensitive to outliers.



\section{The mathematics behind large margin classification}

\screenshot{vectorinnerproduct}{The vedto rinner product}

We use  inner products  \(u^Tv\) and the norm of a vector \(||u|| =\sqrt{u_1^2 + u_1^2}\).  The inner products are positive is the angle
between the vectors is less than ninety degrees, and negative if it is
larger (you know what I mean).

\screenshot{svmdecisionboundary}{The decision boundary for a support
  vector machine}
\screenshot{svmdecisionboundary2}{The decision boundary for a support
  vector machine}
\screenshot{svmdecisionboundary3}{The decision boundary for a support
  vector machine}

The decision boundary (for an appropriate value of C I suppose) is:

\[
  \mbox{min}_\theta \frac{1}{2} \sum_{j=1}^n \theta_j^2 =
  \frac{1}{2} (\theta_1^2 + \theta_2^2) =
  \frac{1}{2}(\sqrt{\parens{\theta_1^2 + \theta_2^2}^2} =   \frac{1}{2} ||\theta||^2
\]
\mXXX{This doesn't look right?}

The parameter vector \(\theta\) will be at right angle with the
decision boundary.   The projections of the sample set down onto the
parameter vector.  The mechanics of it is that by having the decision
boundary as much as possible at right angles to the objects, that
means that the projection down onto the \(\theta\) can be large.
Since the  decision boundary is defined by \(p^{(i)} ||\theta||\) this
means that \(\theta\) can be small, and since our penalty function
contains the sum \(\sum_{j=1}^n \theta_j^2 = ||\theta||\), that means
that we can get a decision with a minimally small \(\theta\)
\mXXX{That at least made sense when I wrote it.}.



\section{Kernels I}

\screenshot{nonlinearboundary}{A nonlinear boundary using a nonlinear kernel}

We'll use kernels to produce complex non-linear classifiers.   We can
use complex polynomial features. A question is how to find better
features than polynomial terms?   One idea is to define new features
using \idx{kernels}.   The idea is to start with a bunch of
\idx{landmarks} \(l^{(1)},l^{(2)},l^{(3)} \).  We will now define the
features to be some similarity between the landmark and the sample:

\[
f_1 = \mbox{similarity}(x, l^{(1)}) = \exp\parens{- \frac{||x - l^{(1)}||^2}{2 \sigma^2}}
\]

The similarity function is called a \idx{kernel} and this particular
kernel is called an \idx{Gaussian kernel}.  The kernel is sometimes
denoted as \(k(x, l^{(i)})\) (ignoring the intercept term).

For a Gaussian kernel, if the x is close to the landmark, then the
euclidian distance in the kernel wil be close to zero, so the feature
will be approximately 1.  Conversely, if the x is far from the
landmark, the feature will be close to zero.


\screenshot{landmarkexample}{How does a Gaussian look with some
  different parameters?}

The feature measures how close something is to the landmark.  We can
modify the \(\sigma\) to make the kernel ``sharper'' i.e. give a
larger penalty for being large distance from the landmarks.  A larger
\(\sigma\) gives less penalty, and a smaller \(\sigma\) gives larger
penalty (the probability distributions has a maximum at 1 and w width
proportional to \(\sigma\).

\screenshot{gaussianprediction}{Using Gaussian to learn complex
  decision boundaries}

Using this Gaussian kernel we can learn pretty complex decision
boundaries.

How do we choose landmarks, and which other similarity functions can
we use?  In the next section we'll learn about that, and how we can
combine it in a support vector machine that can learn complex decision boundaries.

\section{Kernels II}

We'll se how to use kernels and how to do bias/variance tradeoffs.   

\subsection{How to find landmarks?}

In practice, we have some positive and negative examples.  For every
training example we have, we are going to put landmarks exactly at the
location of the training examples.  This gives us \(m\) landmarks (one
for each training examples).  This will give us a classifier that
measures how close things are to our training  set.\mXXX{For a class?}


For each sample we will then get a feature vector.  For a training
example \((x^{(i)},x^{(i)})\)  we will get a feature vector:
\[
\brackets{
\begin{array}{l}
f_1^{(i)}\\f_1^{(i)}\\\ldots\\f_m^{(i)}
\end{array} 
} =
\brackets{
\begin{array}{l}
\mbox{sim}(x^{(i)}, l_m^{(1)}\\
\mbox{sim}(x^{(i)}, l_m^{(2)}\\
\vdots \\
\mbox{sim}(x^{(i)}, l_m^{(m)}\\
\end{array} 
}
\]

\screenshot{svmwithkernels}{Support vector machines with kernels}

We can now plug this into the SVM.  We will be working on a feature
vector that gives us a  similarity to the individual points in the
training set.  We will then be useing a minimization functinon

Not:
\[
  C \mbox{min}_\theta \brackets{
 -(y^{(i)}\mbox{cost}_1(\theta^Tx^{(i)})) + (1 - y^{(i)} ) \mbox{cost}_0(\theta^Tx^{(i)})}
+ \frac{1}{2} \sum_{j=1}^n \theta_j^2
\]

But:
\[
  C \mbox{min}_\theta \brackets{
 -(y^{(i)}\mbox{cost}_1(\theta^Tf^{(i)})) + (1 - y^{(i)} ) \mbox{cost}_0(\theta^Tf^{(i)})}
+ \frac{1}{2} \sum_{j=1}^n \theta_j^2
\]

In this case it is also true that \(m=n\).     There is one
mathematical aside here.  We see trivially that

\[
 \sum_{j=1}^n \theta_j^2 = \theta^T\theta
\]

However, in most implementations of SVMs the actual term used is:

\[
\theta^T M \theta
\]

This is a slightly different distance metric where \(M\) is dependent
on the kernel we use.   A rescaled version of the parameter vector
\(\theta\).  The reason this is done is that we can use much bigger
training sets.   If \(m=10000\) which means that \(\theta\) becomes
really big.  Solving for all of these parameters becomes expensive.
It does change the optimziation criterion as well, but that doesn't
matter very much.

We can in fact use the kernel trick for other classifiers as well,
including logistic regression but the efficiency we can get from the
SVM doesn't translate to other classifiers.   SVNs and kernels are a
match made in heaven (math made in heaven? :-)

It'a recommended to not implement the SVN machine yourself, it's
better to use a library someone else has sweated over first :-) (same
as inverting a matrix or computing a square root).

\subsection{Bias and variance management}

\screenshot{svmparameters}{Parameters for the SVN algorithm}

\(C \approx \frac{1}{\lambda}\) large C, lower \idx{bias} high \idx{variance}
(analogous with small \(\lambda\)).

Small C, higher bias, low variance (analogous with large
\(\lambda\)).  Small values of \(C\) more prone to \idx{overfitting}, large
values of \(C\) more prone to \idx{underfitting}.

The other parameter we need to choose is \(\sigma^2\):

 Large \(\sigma^2\) implies that features \(f_i\) varies more
 smoothly.  Higher bias, lower variance.  Wide Gaussian. More prone to underfitting.


Small \(\sigma^2\) implies that features \(f_i\) varies more abruptly.
Lower bias,  higher variance.  narrow Gaussian. More prone to overfitting.

\section{Using an SVN}

How to practically solve SVN optimization software.  In essence: Use a
library (liblinear, libsvm), there are many other for many languages.

However, there are some things that needs to be done:   
\begin{itemize}
\item   Choose a a parameter \(C\).
\item   Choose a kernel.  
\begin{itemize}
  \item One choice is no kernel, a.k.a. \idx{linear kernel}: ``predict
    y=1'' if \(\theta^T\geq 0\). (``an SVM with a linear kernel'').
    This is the version of SVM that just gives you an ordinary linar
    classifiser. Many libraries can do this.   It is a reasonable
    choice if you have a huge number of features and a small number of
    items in your  training set.  
  \item Another choice is the \idx{Gaussian kernel}: \[
        f_i = \exp{- \frac{||x - l^{(i)}||}{2\sigma^2}}
\]
  where \(l^{(i)} = x^{(i)}\) .  In this case we need to choose
  \(\sigma^2\) as a parameter.

A Gaussian kernel is reasonable if \(x\in\bb{R}^n\), where \(n\) is
amall and \(m\) ( the number of training samples) is large.   E.g. a
two dimensional training set with a nonlinear boundary.
\end{itemize}
\end{itemize}

If you choose  to use a Gaussian kernel, you need to provide the
kernel.   Something of this nature:

\begin{verbatim}
function f  = kernel(x1, 2)
    f = exp(- (norm(x1 - x2)^2 / (s*sigma^2)))
return
\end{verbatim} 

x1 is a test example, x2 is a landmark. The return value f is a real
number.  It's important to perform feature scaling before
using the Gaussian kernel.   If the features take on very different
values, the difference \(x_1 - l_1\) might be much smaller than \(x_2
- l_2\)  which would mean that the \(x_1\) parameters wouldn't be
given any attention.
Some SVM implementations provides both the linear and Gaussian kernels.

Not all similarity functions make valid kernels.  They actually need
to satisfy a condition called \idx{Mercer's theorem} to make sure
that the optimizations run correctly and do not diverge.  This is
actually a design decision that was done when SVMs were
specified\mXXX{Very clever ·-)}.   Some other kernels are:

\begin{itemize}
\item \idx{The polynomial kernel}  \[ k(x,l) = (x^T l)^2\]   This is a
  slightly unusual kernel, but it is used sometimes.  Some other
  examples are \((x^T l)^3\) , \((x^T l+1)^3\) , \((x^T l+5)^4\) .
  There are a couple of parameters: The more general form is 
\((x^T l  + \mbox{constant})^{\mbox{degree}} \) .

  The polynomial kernel almost always performs worse than the Gaussian
  kernel. Always use it when  all the parameters are strictly
  non-negative, so there can never be negative similarities

\item Other kernels (more esoteric): String kernel (text input data), chi-square kernel,
  histogram intersection kernel etc.

\end{itemize}



\subsection{Multi-class classification with SVM}

Many svm packages already have multiclass classification.   Otherwise
one can use the \idx{one-vs.-all method}.   Make \(k\) svms, one for
each class, pick the one with the largest \((\theta^{(i)})^T x\).

\subsection{Logistic regression v.s. SVMs}

If n is number of features and m number of training examples.

If \(n\geq m\), e.g. in a text classificaiton problem then use
logistic regression or SVM without a kernel.  A linear function will
do fine, and we don't have enough data to fit a fancy kernel.

if n is in the  range from 1-10000 and m is in the range 10-10000 (but
not a million), then use SVM with the Gaussian kernel.

If n is small but m is large (50K, a million or large), then create
more features, then use logistic regression without a kernel.

Logistic regression and SVM without a kernel will usually do pretty
similar things and get similar performance.  If one works, then the
other will probably work pretty well too.

With about ten thousand  to fifty thousand learning examples SVMs with
gaussian kernels really shine.

Neural networks are likely to work well for most of these settings,
but may be slower to train.  SVMs can be very much faster.

Also the SVM packages have a convex optimization problem, so we will
always find the global optimum.

These guidelines are a bit vague, but that's ok.     The algorithm
does matter, but error analysis, the size of the data set, getting the
right features and those things often mattes more.   SVMs are still
conidered to be very efficient.   With logistic regression SVM, neural
networks are basically the state of the art of machine learning
systems for a wide range of problems.

\chapter{Unsupervised learning: Clustering}

The problem is that we are given data without any labels to it, so our
training set is just a bunch of data.   In unsupervised learning we
give this type of data to an algorithm and asks it to find some
structure.   One type of structuring is \idx{clustering}, but there
are others.   

There are many applications of  unsupervised learning:  Market
segmentation, social network analysis, organizing computing clusters
(if you know which computers work tightly together, you can optimize layout).
and astronomical data analysis.

\section{K-means algorithm}


We would like to cluster data into coherent clusters.  K-means is by
far the most used clustering algorithm.  The algorithm goes like this:

% This should be an image array
\screenshot{kmeansclustering1}{K means clustering}

\screenshot{kmeansclustering2}{K means clustering algorithm}
\screenshot{kmeansclustering3}{K means clustering algorithm}

\begin{itemize}
\item first randomly select the \idx{cluster centroids} (as many or few as you
  like, the algorithm will not determine the optimal number of
  clusters for you).  
\item Cluster assignment step:   Assign the datapoints to the centroid
  that is the closest.  (use a voronoi tessalation of the plane )
\item Move centroid step: Move the centroids to the average (mean)
  location of all the points in the cluster.

\item Go back to the cluster assignment step.

\item Repeat until convergence.

\end{itemize}

That was it :-)  K-means does a pretty good. We'll drop the intercept
term.  Upper case K is used to indicate the number of cluster
centroids, lowercase k used to indicate a particular cluster
center. The squared distance is used by custom (minimizing square or
unsquared doesn't change the result of the algorithm).

If you get one cluster centroid you can either eliminate that centroid
(the ordinary thing to do), or just randomly place the centroid.

\screenshot{nonseparatedkmeans}{One can cluster also when the data
  isn't clearly ``clustered'', for instance the graph on the right
  might be useful for clustering T-Shirt shizes}

K-means for non-separated clusters, it sometimes works well like in
the T-shirt example in figure \ref{nonseparatedkmeans}.  You get a
market segmentation out of the box. It may even be useful.

\subsection{Clustering optimization objective}


Clustering also has a cost function.  This is useful because it helps
us debug the algorithm, but it can also help us to find better
clustering and avoid local optima.
\begin{tabular}{lcl}
\(c^{(i)}\) &=&a cluster of (\(1,2, \ldots, K\) to which example
clustering and avoid local optima\(x^{(i)}\) is currently assigned. \\
\(\mu_k\) &=& cluster centroid \(k\) (\(\mu_k \in \bb{R}^n\)) \\
\(\mu_c(i)\) &=& Cluster centroid of cluster wo which example
clustering and avoid local optima\(x^{(i)}\) has been assigned\\
\end{tabular}


\screenshot{kmeanscriterion}{The optimization criterion for K-Means clustering}

With this we are ready to type out the optimization objective for the
clustering and avoid local optimak-means algorithm, and in figure
clustering and avoid local optima\ref{kmeanscriterion} it is.

The average of the squared distance of the distance between the
training examples and the centers of the clusters they are assigned

\screenshot{kmeansalgorithm}{The K-Means algorithm as pseudocode}

The cost function is called the \idx{distortion function}.  In a
sense what the k-means algorithm does is to partition the parameters
and then use the assignment step to minimize one part (the cluster
assignments) and then to minimize the other part (the assignments),
and then repeat until convergence.

\subsection{Random initialization}

There is one highly recommended way to pick cluster centroids.  First,
let \(K < m\).  Then randomly pick \(K\) training examples and let
\(\mu_1,\ldots,\mu_K\) be equal to these \(K\) samples.  Easy peasy.

\screenshot{localoptimakmeans}{localoptimakmeans}

K-means can end up at different solutions depending on which centroids
we start at (local optima of the distortion function).  To avoid this,
we can try multiple random initialization to make sure that we get as
good a global optimum as possible. Perhaps fifty to hundred times.  We
will then {\em compute the distortion} for each of the candidates.  We
then pick the clustering with the lowest cost.

It turns out that if the number of cluster is small (to to ten) can
sometimes find a better optimum, but if K is large then random
initilization will improve the solution since the initial solution
will be pretty good.

\section{Choosing the number of clusters}

There isn't a good way to do this automatically.  The most common way
to do this is to do it manually. 

It's like this since it is often genuinely ambiguous how many clusters
there are.  There is an automatic method called the \idx{elbow method
  of cluster number determination}.

\screenshot{elbowmethod}{elbowmethod}

The idea is that plot of the cost and the number of clusters, and then
look for an ``elbow'' where the drop in the error function seems to be
leveling out.  That's where the cutoff point is placed.

\screenshot{elbowmethod}{The ``Elbow'' method for choosing the number
  of clusters looks at how the cost function varies with the number of
cluster.  If an ``elbow'' pattern can be observed, it can be used to
select the appropriate number of clusters, but often there is no clear
elbow and then the method is not applicable.}

When it works and you really get an ``elbow'', use it and be happy.
However, it can often not work.  It's worth a shot, but it doesn't
actually have a high expectation of giving a clear cut answer for any
particular problem.

Another way to choose K to use some other metric. E.g. how well the
downstream method (market segmentation, computer cluster configuration
or something) fares.  If there is an evaluation method available in
the downstream problem, one should consider using that.

\screenshot{tshirtclustersize}{Choosing the number of clusters is
  based on some metric on how well some downstream metric matches the
  number you choose.  How many sizes of T-Shirt to produce may or may
  not be directly dictated by the variation in the population of
  T-shirt consumers.}

For instance, we could have three or five sizes of T-shirts
(fig. \ref{tshirtclustersize}).)  This is actually a business-question
that isn't obvious from the cluster metric.

Tradeoffs between filesize and compression distortion in a k-means
cluster based image compression method  is another example of using
the downstream evaluation method.

\chapter{Dimensionality reduction}

DImensionality reduction is another type of unsupervised learning.
We can use it for compression but will also help us to improve the
performance of our learning algorithms.

If you have a bunch of features (several thousand from different
sources) so it's common to have highly redundant parameters.
(measuring inches and cm, for instance).   A couple of other things that might by
highly correlated is pilot enjoyment and pilot skill, so instead one
could use a common variable to describe them both (and perhaps call it
``pilot aptitude'', although it isn't necessary to give these things
explicit names in order to make use of them).

\screenshot{cpaprojection}{PCA Principal component analysis, is used
  to find the directions in the dataset where the variation is
  maximal.   Projecting the data points along these directions
  (vectors) will reduce the number of coordinates that are necessary
  to describe the dataset with high accuracy.}

We approximate the original dataset by projecting down on the
synthesized feature vector.  


We can also compress data from 3D to 2D (or ten thousand dimensions to
a hundred dimensions, but that's harder to show on a screen)

\screenshot{3dcompression}{The three dimensional data in this dataset
  all lie mostly in a plane. The points in this plane can be described
by being projected on two vectors in the plane, and thus only two, not
three coordinates are necessary to describe most of the variation in
the dataset.}

In fig \ref{3dcompression} the original 3D data is located in a plane
( a subspace of 3D), ao we can project all of the data down on that
plane, and get 2D coordinates instead of 3D coordinates.

\subsection{Visualization}

\screenshot{worldddata}{Gdp v.s. the size of the country}

Visualization is a really nice way to help us understand the data we
use.  If we look at the data in fig \ref{worldddata} it's difficult to
see any kind of structure in that data, at least at first glance.  If
there are fifty or more data per country the problem gets worse :-)

Using dimensionality reduction we can project this data down onto a
lower dimensional space with e.g. two or three dimensions.  If that
representation can summarize the fifty numbers, then we can plot the
data and understand the variation.

\screenshot{worlddata2d}{worlddata2d}

Usually the new dimensions must be interpreted by us (humans). We
typically get something like what is shown in fig \ref{worlddata2d},
but we must figure out what the dimensions really mean.

\section{Principal component analysis}

PCA  minimizes the sum of squares of the distances from  the dataset
to a surface.  This error is called the \idx{projection error}.   A
standard practice is to perform mean normalization and feature
scaling, so that the features have zero mean and a comparable range of
values.

Formally the goal of PCA is to find a direction onto which to project
the data to minimize the projection error.  It doesn't matter if the
direction of the vector goes in a positive or negative direction,
since the surface containing the vector is the same.

\screenshot{pcaobjective}{The objective of the PCA algorithm is to
  find the directions that minimizes the projection error.}

We're out to project the data onto the linear subspace defined by our
vectors.

Btw, PCA is not linear regression even though there are some cosmetic
similarities.

\screenshot{pcavslinreg}{PCA may superficially look similar to linear
  regression, but they are in fact quite different:   PCA optimizes
  the minimum of the distance to the line, but linear regression
  optimizes the minimum distance between data points in the line only
  in the vertical direction.}

In PCA there is no distinguished ``y'' we are trying to find.  

\subsection{The PCA algorithm}

Before using a training set \(x^{(1)}, x^{(2)},\ldots, x^{(m)},\) it's
important to perform feature scaling/mean normalization.  First find
the mean:

\[
 \mu_j = \frac{1}{m} /sum_{i=1}^m  x_j^{(i)}
\]


Then We should replace each  \(x_j^{(i)}\) with \(x_j - \mu_j\).  If
different features are on different scales (e.g. sizes of houses in
square feet and number of bedrooms),  it will be hard to compare
values.  This is the same process used for supervised learning.

\[
   x_j^{(i)} \leftarrow \frac{ x_j^{(i)} - \mu_j}{s_j}
\]

where \(s_j\) is the maximum, minimum or most commonly standard
deviation of the \(x_j\) values.

The procedure is pretty simple:

\begin{itemize}
\item Compute the \idx{covariance matrix}
\[
    \Sigma =\frac{1}{m} \sum_{i=1}^{n} (x^{(i)}) (x^{(i)})^T = \frac{1}{m} X^TX
\]

\item Compute the \idx{eigenvectors} of the matrix \(\Sigma\):
\begin{verbatim}
  [U,S,V] = svd(Sigma);
\end{verbatim}
\end{itemize}
 
The difference between ``svd'' and ``eig'', but ``svd'' is a bit more
numerically stable, but applied to a \idx{covariance matrix} it will always
give the same answer.  All covariance matrices are \idx{symmetric
  positive semidefinite} but that's not important right now :-) That
wasn't so hard was it?

The covariance matrix will be an n by n matrix.   The U matrix is an N
by N matrix containing columns that are the vectors we want to project
down onto. If we want to project only onto k vectors, then we pick the
k first vectors (columns) in the U matrix.  We stack these columns and
call  this n by k matrix the \(U_{\mbox{reduce}}\) matrix.  We can then
compute the recued data set \(Z\) like this:

\screenshot{pcacorestep}{Restoring data from PCA}

\screenshot{pcaonsider}{PCA in a single slide. The SVD function in
  Octave gives us what we need (not optimally efficient but it's ok)}
                   
\[
  Z = U_{\mbox{reduce}}^T X
\]


\section{Choosing the number of dimensions to extract using PCA}

The parameter \(k\), the number of \idx{principal components} that we
are extracting using PCA is something we must determine.    Here are
some ways to do that.   There are some things that are of interest

\begin{itemize}
\item Average squared projection error: \[\frac{1}{m} \sum_{i=1}^m  ||x^{(i)} - x_{\mbox{approx}}^{(i)}||^2\]
\item The total variation in the data: \[\frac{1}{m} \sum_{i=1}^m  ||x^{(i)}||^2\]
\item The typical rule to use:  Retain 99 percent of the variation:

\[
\frac{\frac{1}{m} \sum_{i=1}^m  ||x^{(i)} -
  x_{\mbox{approx}}^{(i)}||^2}{\frac{1}{m} \sum_{i=1}^m  ||x^{(i)}||^2}
\leq 0.01
\]
Some other values would be 0.05 (five perdent of the variance
retained) etc.  Perhaps as low as 0.85, but 95-99 perent is useful.
For many real data datasets it is often easy to get these numbers with
relatively small values of \(k\).
\end{itemize}


One algorithm is to perform PCA on the training set until the variance
target is reached and then select the \(k\) that gave the wished-for
number.  However this is inefficient.  Fortunately PCA gives us a
value that can be used.   The inner-loop gives us these results.

\begin{verbatim}
 [U, S, V] = svd(Sigma)
\end{verbatim}


\xscreenshot{choosingk}{choosingk}

The \(S\) value is a diagonal matrix of \idx{eigenvalues}
(\idx{singular values} in this case), and it so happens that we can
use that to compute the quantity describing the retained variation
much more simply than the hairy expression above.   Explicitly,
assuming that the \(S\) matrix has nonzero elements (\(s_{11}, s_{22}, \ldots,
s_{nn}\)) then 

\[
\frac{\frac{1}{m} \sum_{i=1}^m  ||x^{(i)} -
  x_{\mbox{approx}}^{(i)}||^2}{\frac{1}{m} \sum_{i=1}^m  ||x^{(i)}||^2}
= 1 - \frac{\sum_{i=1}^k s_{ii}}{\sum_{i=1}^n s_{ii}}
\]

And (assuming that calculating all the eigenvalues is efficient), then
this will be an efficient method.

Plotting the \(s_{ii}\) values in  a histogram could also be used in
the ``\idx{elbow}'' method described in a previous section to see if
there are any kinks in the value of variance retained vs k and select
the number where the explained variance levels off. \mXXX{Actually Ng
  doesn't recommend this, but imho  it's still a good idea if the
  point is to avoid overfitting, which really isn't so  much of an issue
  if compression is the point :-) }

\section{Reconstruction from compressed representation}

How to go back from the compressed data back to the original
representation?

Consider  a two dimensional dataset we wish to reduce.   We do that by
projecting:

\[
         z = U^T{\mbox{reduce}} x
\]

If we want to go the oposite direction:

\[
   x_{\mbox{approx}} = U_{\mbox{reduce}} z
\]

This process is called \idx{reconstruction} of the original data.

\section{Advice for applying PCA}

PCA can be used to speed up a supervised learning algorithm.  This is
the most common way that Ng uses PCA.  Assume that we have a very high
(ten thousand or so) feature vectors (e.g. in computer vision):



\[
 (x^{(1)}, y^{(1)}),  (x^{(1)}, y^{(2)}),\ldots,  (x^{(m)}, y^{(m)})
\]


Assume that we wish to use some algorithm, \idx{SVM}, \idx{neural
  network} or whatever we can reduce the algorithm to work on lower
dimensional data which will make it run more efficient.  The method to
use is illustrated in figure \ref{applyingpca}.  Just remember to
apply the \(U_{\mbox{reduce}}\) found in the training set on the cross
validataion and test sets.  It is often possible to reduce the number
of dimension five or ten times without reducing the classification
performance.   When using PCA for learning algorithms we are using a
variance retention metric and use some high k.

For visualization we often use k=2 or 3.

\subsubsection{Misuse}

There is a frequent misuse of PCA: To prevent overfitting.  The number
of dimensions (features) in the learning algorithm, fewer data will
prevent overfitting.    This is {\bf bad}.   Use regularization
instead :-)  PCA throws away data without regards to the \(y\) value
and will often perform worse than regularization, which knows
something of the value of \(y\).


\xscreenshot{pcaplanwithflaw}{pcaplanwithflaw}

Before committing to using PCA, one should also consider doing the
whole thing without PCA:  PCA adds complication and if the project is
feasible without PCA, then that's perhaps the best way of doing it.

Only use PCA if the algorithm runs too slow, or the memory footprint
is too large.

\chapter{Anomaly detection}

AD has some features of unsupervised learning but have som aspects of
supervised learning. Assume we are a manefacturer of aircraft
engines.  We measure features (heat, vibration etc.)  You now have a
data set of many features.  The anomaly detection problem is to to
check if any engine is anomalous in any way, should it undergo further
testing or something.   Anomalies should be examined more before it
sent to the customers.

\screenshot{densityestimation}{Using Gaussians to estimtate the
  density of a set of data points}

Formally, if we see a sample that has a very small probability of
being part of the model that defines the normal samples, then we
should flag an anomaly.  Some examples of anomaly detection is

Fraud detection (user activities, with a model of user activities).
Identify unusual users by checking which have \(p(x)<\epsilon\).
(typing speed, etc.).

This flags users that behave strangely, not just fraudsters.


Another example is manufacturing.  

Yet one is montiing computers in a data center (memory use, numer of
disk acccesses, CPU load,  load /network traffic, etc.)

\section{The Gaussian distribution}

\screenshot{gaussiandistribution}{The Gaussian distribution and the
  standar parameters used to describe it.}

\[
   x \sim \mathcal{N}(\mu, \sigma^2)
\]

The Gaussian probability distribution is parametrized has mean \(\mu\)
and with defined by \(\sigma\).

\[
    p(x;\mu,sigma^2) = \frac{1}{\sqrt{2\pi}\sigma}\exp{\parens{-\frac{(x-\mu)^2}{x\sigma^2}}}
\]

\(\sigma\) is called the \idx{standard deviation} and \(\sigma^2\) is
called the \idx{variance}.

\screenshot{gaussianexamples}{Some examples (wide and narrow) of
  Gaussian distributions and their parameters.}

The \idx{parameter estimation} is the problem of fitting a data set to
a Gaussian.  How to find the parameters \(\sigma^2\) and \(\mu\)for
the Gaussian that best fits the given data?  The standard formulas for
making these estimates are:
\screenshot{gaussianestimation}{Fitting a data set to a Gaussian is
  easy, just find the average and the variance and you're done.}

\[
\begin{array}{lcl}
\mu         &=& \frac{1}{m} \sum_{i=1}^{m}x^{(i)} \\
\sigma^2 &=& \frac{1}{m} \sum_{i=1}^{m}\parens{x^{(i)} - \mu}^2 \\
\end{array}
\]

These parameters are actually the \idx{maximum likelyhood estimators}
for the normal distribution.  Note (some versions of this formula use 1/(m-1)).


\section{An anomaly detection algorithm}

Assume we have a training set \(x^{(i)}, \ldots, x^{(m)}\), each
sample is a real number.  We'll model this as a product of Gaussian
distributed random variables, so that \(x_i \sim mathcal{N}(\mu, \sigma^2)\).

\[
   p(x) = \prod_j^n p(x_j;\mu_j,\sigma_j^2)
\]

This equation correspond to an independence assumption on the
features, but it usually works fine even if the terms are not entirely
independent.  This problem is also called the problem of \idx{density
  estimation}.

\screenshot{anomalydetectionalgorithm}{An algorithm to detect
  anomalities: Find the multidimensional Gaussian, see how well a
  single datum is described by this Gaussian, and establish a
  threshold to define anomalities.}

An algorithm to detect anomalies can then be:

\begin{itemize}
\item Choose features \(x_i\) that might be indictive of anomalous
  examples (e.g. temperature, vibration, speed of typing etc.)

\item Fit the parameters (\(\mu_1, \ldots, \mu_n, \sigma_1^2, \ldots
  \sigma_n^2\)) using the max likelyhood estimator:

\[
\begin{array}{lcl}
\mu         &=& \frac{1}{m} \sum_{i=1}^{m}x^{(i)} \\
\sigma^2 &=& \frac{1}{m} \sum_{i=1}^{m}\parens{x^{(i)} - \mu}^2 \\
\end{array}
\]

\item Given a new example \(x\), compute \(p(x)\) using:

\[
p(x) = \prod_{j=1}^n p(x_j; \mu_j, \sigma_j^2)=\prod_{j=1}^n\frac{1}{\sqrt{2\pi}\sigma_j}\exp{\parens{-\frac{(x-\mu_j)^2}{x\sigma_j^2}}}     
\]

Assume that you're looking at an anomaly if \(p(x)<\epsilon\).
\end{itemize}


The algorithm can be vectorized, e.g. using this trick:

\[
 \mu = \brackets{
\begin{array}{c}
\mu_1\\\mu_2\\\vdots\\\mu_n
\end{array}} =
\frac{1}{m}\sum_{i=1}^m x^{(i)}
\]

\screenshot{anomalyexample}{Example of an anomaly detector}

\section{Developing an evaluating an anomaly detection system}

It's important to have real numbers that can be used to evaluate the
learning algorithm, this is called \idx{real-number evaluation}.  We
can use this to include/exclude features etc.c We will assume that we
have som labelled data, assuming \(y=0\) if normal and \(y=1\) as
anomalous.  We then take a training example which is a large example
of normal samples, however it's also ok to have a few  anomalous cases
too. We then have a cross validation set and a test set. 

Assume 10000 good (normal) engines.  We have 20 flawed engines in this
training set. Usually there are much more good than bad samples.  One
way of splitting is to use 6000 known good as the unlabled training
set (y=0), test is 2000 engines with ten known anomalities and 2000
(also with ten known anomalities) is used for crossvalidation (6+2+2).
Using the same items in the crossvalidation and test sets is not
usually a good idea.

We then evaluate the algorithm like this:

\begin{itemize}

\item Fit a model on the training set (max likelyhood estimation of
  the Gaussians).

\item On the cross-validation test examples \(x\), predict:

\[
y = \left\{  
\begin{array}{lclcll}
1 &\mbox{if}& p(x) &<& \epsilon &\mbox{anomaly} \\
0 &\mbox{if}& p(x) &\geq& \epsilon &\mbox{anomaly} \\
\end{array}
\right.
\]
\item Possible  evaluation metrics:

\begin{itemize}
\item True positives, false positives, false negatives, true negatives
\item Precision/recall
\item \(F_1\)-score  (a way to summarize preciesion/recall things)
\end{itemize}

\item We can also use cross validation to choose the parameter
  \(\epsilon\).  Just try many and pick one that maximizes \(F_1\) or something.

\end{itemize}

This is actually not that dissimilar to supervised learning.  The
labels will be very \idx{skewed} since there are many more of the
normals than the anomalies.  Classification accuracy is not very
accurate.

We can then use the set of parameters and evaluate the parameter use
on the cross validation set.

\section{Anomaly detection v.s. supervised learning}

If we have labelled data, why don't we use a supervised learning
algorithm?  You should use anomaly detection when very small number of
positive examples and a large number of negative examples.  We'll save
the negative examples for test and cross validation.  If you have lots
of both positive and negative examples so then you can use supervised
learning.

Another way is to detect many different ways things  things can go
wrong.  It's hard to learn from the small set of things that went bad,
in particular the future anomalies may look  nothing like the
anomalous examples seen so far.

On the other hand, given enough examples of both classes  it's
possible to get a good sense of what future bogusness will look at.   

All of this is just another way of saying that anomaly detection is
usually better when there are very \idx{skewed samples}.

Some applications of anomaly detection: Fraud detection, manufacturing
QA, Monitoring machines in data centers.   If you have a lot of
examples, sometimes supervised learning can be used.

Supervised learning (spam classification, weather prediction, cancer
classification (plenty of samples).  

In many types of settings there are actually zero samples of negative
samples, and in those cases AD is usually used.

\section{Multivariate Gaussian distribution}

\screenshot{datacenterexample}{Memory use v.s. machine load in a data
  center: What is an anomalous situation?}

MGD has some advantages over the earlier algorithm since it makes use
of covariance structures in the data being used.   The previous
univariate method did not make use of these structures.   In  the data
center example (see \ref{datacentreexample}), the univariate algorithm
will not flag an anomaly even though one should have been flagged.

\newcommand{\mathbb}[1]{{\bf #1}}
The modified version will not model the individual parameters
directly, but we'll model \(p(x)\) for \(x\in \mathbb{R}^n\)
modelling the same thing in one go. The input parameters is the
multivariate average \(\mu\)  and the \idx{covariance matrix} \(\Sigma \in
\mathbb{R}^{n\times n}\).


\begin{verbatim}
% Bogus mathaccents, but why?
\[
p(x;\mu, \Sigma) = \frac{1}{(2\pi)^\frac{n}{2}
  |\Sigma|^\frac{1}{2}}\exp\parens{-\frac{1}{2}\parens{x - \mu}^T \Sigma^{-1}(x-\u)}
\]
\end{verbatim}

\(|\Sigma|\) is the \idx{determinant} of \(\Sigma\), can be computed
using the octave command \verb!det(Sigma)!.  Some examples of
multivariate Guassian distributions can be admired in figure \ref{multivargaussexamples}.

\screenshot{multivariategaussexamples}{Examples of multivariate
  Gaussians with parameters.}

\screenshot{offidagonalsgaussians}{Gaussians where the variation is
  not along the horizontal or vertical axes.}

\screenshot{translatedgaussians}{A Gaussian where the center is translated}


smaller \(\Sigma\) gives sharper bump.  Increased sigma gives blunter
bump :-)  In fact, we can get rotations giving correspondences that
are off-diagonal.  This means that we can capture both negative and
positive correlations.


By varying the mean (the \(\mu\) parameter) we can translate the peak
of the distribution to various points (see \ref{translatedgaussians}).

The key advantage of using the multivariate Gaussian disribution is
that it actually captures these covariations and makes ``sharper'' filters.

\section{Anomaly detection using the multivariate gaussian distribution}

\screenshot{multivariateparameterfitting}{Fitting parameters to a
  multivariate Gaussian.}

\screenshot{multiparamalgorithm}{multiparamalgorithm}{An algorithm to
  check how well a data point fits with a multidimensional Gaussian}


We see that \(\Sigma = \frac{1}{m} X^TX\) after subtracting the mean.

The algorithm is

\begin{enumerate}
\item Fit the model \(p(x)\) by setting:
\[
\begin{array}{lcl}
\mu&=&\frac{1}{m} \sum_{i=1}^{m}x^{(i)} \\
\Sigma&=&\frac{1}{m} \sum_{i=1}^{m}(x^{(i)-\mu})\cdot(x^{(i)-\mu})^T \\
\end{array}
\]
\item Given a new example \(x\) compute:

\[
p(x) = \frac{1}{(2\pi) ^{\frac{n}{2}} |\Sigma|^{\frac{n}{2}}}
\exp\parens{-\frac{1}{2} (x - \mu)^T\sigma^{-1}(x - \mu)}
\]

flag anomaly if \(p(x) < \epsilon\).



\end{enumerate}

\subsection{Relationship with the original model}

\screenshot{newandolldgaussianmodel}{The connection between a
  multivariate Gaussian and univatriate Gaussians (at least that's
  what I believe this plot is all about)}



The original model are actually a special case of multivariate
gaussians where the contours of the Gaussians are always \idx{axis
  aligned}, never at an angle.  The defining property for the original
model is that has only zero elements on the off-diagonal elements of
the covariance matrix.

When would you use these models?

The original model is used more often,  and the multivariate gaussian
is less often used.

In the original model you can create extra features (such as
\(x_1/x_2\) features to capture  unusual combinations.  The
multivariate version captures these automatically.   

The original model is computationally cheaper, so it scales to high n
(tens or hudred thousands).  The multivariate version needs to invert
a matrix, which scales much less well.

The original version works well with a small training set.  The
multivariate version needs \(m>n\) otherwise \(\Sigma\) will be
non-invertible (singular). (some changes can be used :-)

Rule of thumb,  use it only if m is much larger than n (ten times or
so).  There is a lot of parameters (\(\Sigma \sim \frac{n^2}{2}\)).

One technical property of the multivariate gaussian method is that the
\(\Sigma\) may be \idx{singular}, there is usually two reasons for
this:  Either \(n<m\) or redundant features (like \(x_1 = x_2\)).
First check the m and n values then for redundant features.  (\idx{linearly
depenent features}).

\chapter{Recommender systems}

Recommender systems is an important application of machine
learning. It's a hot topic in Silicon Valley right now :) (netflix,
amazon etc.) Performance on recommender systems give immediate
feedback to the bottom line.

Reommender systems is not very big part of what happens in the machine
learning seen from academia. Another thing is that the features that
you choose.  For some types of prolems there are algorithms that is
pretty good at choose which features should be used.

To motivate the problem we will be using the problem of predicting
movie rating.  Movies can be rated from one to five stars (zero is ok
in this case since that lets the makes the math easier)

\screenshot{movietablaux}{A small table of movies and individual
  users' preferences}

Alice and Bob seems to like the same type of movies, and so does Carol
and Dave.   We get rating info, but that rating isn't complete, so we
have two arrays to represent both the presence and the value of
ratings.

\mXXX{Should add the content of the figure as tex here}

The job in recommender systems is to figure out how to find
interesting things for an user.

\section{Content based recommendations}

\screenshot{contentbasedrecommender}{A Content based recommender}


In conent-based recommendations we add a bunch of features, in fig
\ref{contentbasedrecommender} we add ``romance'' and ``action''.  With
features like this, the movies can be represented as feature vectors.
We add a feature always equals to one \(x_0=1\) as an \idx{intercept
  term}.

\(n=2\) since we have two features, nd we don't count the intercept
term.  We could treat the star-prediction problem as a linear
regression problem and let it rip.

One formulation is:
\begin{itemize}

\item \(r(i,j) = 1\) if user \(j\) has rated movie \(i\), (0
  otherwise).

\item \(y^{((i,j)}\) = rating by user \(j\) on movie \(i\) (if
  defined)

\item \(\theta^{(j)}\) parameter vector for user \(j\).

\item \(x^{(i)}\) feature vector for movie \(i\)

\item For user \(j\), movie \(i\), predict rating: \((\theta^{(j)})^T  (x^{(i)}) \)

\item \(m^{(j)}\) no of movies rated by user \(j\)

\item to learn \(\theta^{(j)}\) we can solve this using linear
  regression.  Use gradient descent or the normal equations in order
  to minimize:

\[
   \min{\theta^{(j)}}  \sum_{i:r(i,j) = \frac{1}{2 m^{(j)}}
     1} \parens{(\theta^{(j)})^T(x^{(i)}) - y^{(i,j)}}^2
 + \frac{\lambda}{2 m^{(j)}} \sum_{k=1}^{n} \parens{\theta_k^{(j)}}^2
\]

We can add  a \idx{regularization} term after the sum (done in the equation
above).  We don't regularize over the bias term (as usual).
\end{itemize}


In the subsequent math we'll get rid of the constant term \(m^{(j)}\)
since it will not change the optimization result.


\screenshot{recommenderobjective}{An objective for a recommender
  system:   Predict what the user is most likely to do next, and then
suggest that}
\screenshot{gradientdescentrecommender}{Gradient descent for a
  recommender system}


In figure \ref{recommenerobjective} we calculate all of the
objectives for all the users.   We can put this together in a gradient
descent update algorithm as depicted in figure
\ref{gradientdescentrecommender}.  The only difference wrt to linear
regression is the 1/m term that is missing here.  This objective can
be plugged into \idx{conjugate gradient descent} or \idx{LBFJS} instead.

Content based recommendations require a content feature vector.  For
many types of movies we don't have those types of vectors, so we can
use other, non-content based recommencations.

\section{Collaborative filtering}

\screenshot{featuredetection}{Detecting features from user preferences}
\screenshot{collaborativeoptimizationproblem}{Collaborative optimization}


Has feature learning as one of its atributes.   Often it's hard to
discover the features.   The basic assumption is that the users has
given us a vector that tells us what \(\theta^{((j)}\) then we can
infer what the feature vectors are for any given movie.    How can we
find the \(\theta^{((j)}\)s from the data set?

\screenshot{simplecollaborativefilteralgorithm}{A simple collaborative
filter algorithm}


One algorithm is to randomly guess a \(\theta\) to learn features for
the  movies, and then use that to bootstrap the algorithm, and then
iterate iterate.\mXXX{Isn't this an eigenvector finding method?}

The idea is that every user is helping the algorithm a little bit by
letting the system learn features better.

There is an efficient way that doesn't need to go back between \(x\)
and \(\theta\) and solve it all in one go using the optimization
criterion on the bottom of figure \ref{efficientcollaborativefilter}.
Basically the bnottom criterion is the two on top put into a single
criterion and optimized for all the variables.   The first sum on the
bottom has a term that says \((i,j):r(i,j) = 1\) which means that the
inner sum is over all the pairs of \((i,j)\) such that \(r(i,j)\) is
equal to one, which means that user \(i\) has made a rating of movie
\(j\).  It's just a quantifier hooked up to a summation mechanism.

The two regularization terms for \(\theta\)-s and  \(x\)-es.

There is no intercept term in this system either.  This is why we are
learning all the features, because we if we need a feature that is
always equal one, the algorithm will find one for itself.

The collaborative filtering algorithm:

\begin{itemize}

\item Initialize \(x^{(1)}, \ldots, x^{(n_m)},\theta^{(1)}, \ldots ,
  \theta^{(n_u)}\) to small random values (much like we did for neural
  networks).

\item Minimize \(J((x^{(1)}, \ldots, x^{(n_m)},\theta^{(1)}, \ldots ,
  \theta^{(n_u)})\) using gradient  descent, e.g. for every \(j=1,
  \ldots, n_u, i = 1, \ldots, n_m\): do 
\[
\begin{array}{lcl}
   x_k^{(i)} &:=&  x_k^{(i)} - \alpha \parens{\sum_{j:r(i,j) = 1} ((\theta^{(j)})^T x^{(i)} - y^{(i,j)})\theta_k^{(j)} + \lambda x_k^{(i)}}\\
   \theta_k^{(i)} &:=&  \theta_k^{(i)} - \alpha \parens{\sum_{j:r(i,j)= 1} ((\theta^{(j)})^T x^{(i)} - y^{(i,j)}) x_k^{(j)} + \lambda \theta_k^{(i)}}\\
\end{array}
\]

All of the parameters are regularized.  The intercept term is not
present so there is no \(\theta_0\) to treat speicially.

\item For an user with parameter \(theta\) and a movie with learned
  features \(x\) predict a star rating of \(\theta^T x\).
\end{itemize}

This is actually  pretty decent algorithm both for modelling user
behavior and for extracting parameters describing the movies.

\section{Low rank matrix factorization}

\screenshot{lowrankmatrix}{Lower rank matrix? What does this mean?}
\screenshot{predictedratings}{Predicting rating from incomplete data}


To vectorize collaborative filtering we need some tricks we can use.
We can use the matrix Y in figure \ref{predictedratings}. This
vectorized implemntation is called the \idx{low rank matrix
  factorization} due to the fact that the matrix \(X\theta\) is a low
rank matrix.

Finally, having run the collaborative filter we can find related
movies.  It is often hard to figure out what the features exactly are,
but usually the features that are captured will often find the most
salient  features fr figuring out why people like those movies.
We can use this to measure how similar two movies are, we try to
minimize some distance between the movies, we can use this to find
e.g. the five movies that are the most similar to the one the user is
seeing right now.

\subsection{Mean normalization}

\screenshot{userswithoutrating}{How to handle users without any ratings}

The mean normalization will usually let the algorithm run a little bit
better.  If we have an user without any ratings, if we include the
user's \(\theta\) in the expression to minimize, the regularization
term will force all the terms to become zero and that will impose no
penalty so that is what we get.  We will then assume that the user
with no ratings don't like anything.  It seems that we are actually
overfitting to the data we have, so perhaps some sort of smoothing
will help us?  That is where the idea of mean normalization comes into
play.


\xscreenshot{meannormalization}{Implementing mean normalization}

First we calulate the averages of all the rows by first summing up all
the nonzero but defined values, and the divide by the number of
nonzero but defined values.  Then we subract this value from all the
values in the row.  This we do for all the rows, then we have the
``mean'' of the rows subtracted from the row.  We subtract for each
row the average rating for the row.  The question marks stay question
marks.

\screenshot{meannormalizedresult}{meannormalizedresult}

The new mean normalized ratings is then used by the learning
algorithm. This means that we need to add back the mean before
returning the result.  The results actually makes sense.

If we have no movies with no ratings, we can play with version of
the algorithm to let columns get some mean values, but that may not be
such a good idea :-)

\chapter{Large scale machine learning}

Learning with big datasets.  One of the reasons algorithms works
better today than five or ten years ago is that we have massive
datsets, now we will learn how to work with that.  One of the best
method of getting a good algorithm is to use a low-bias algorithm an a
huge learning set.  ``It'w not who has the best algorithm, but who has
the most data that wins''.   

Learning with large datasets comes with computationa cost.  Working
with hundreds of millions of records is very realistic.  If you wish
to use a gradient descent algorithm on a dataset like this, we need to
need to compute a summation over a hundred million entries just to
compute one step of gradient descent.  We'll now learn both methods
for replacing the algorithm, and more efficient ways to compute
derivatives.  We'll learn how to fit neural networks and regression
models.

First we must ask us why not use a subset, e.g. a thousand samples?
This is a good thing to check.   The way to check this is to make plot
the \idx{learning curve}.  If we have a high variance problem and it
may be a good idea to add more samples.  If we have a high bias case
then it is unlikely that increasing the learning set will help us.

However, i a high bias method adding more features or adding more
hidden nodes in a neural networks.  We'll consider two algorithms that
can be used to handle really big datasets, one is \idx{stochastic
  gradient descent} and the other is \idx{map reduce}.


\section{Stochastic gradient descent}

For many learning algorithms we have a cost function  and then use
gradient descent to compute minima.  When we have very many variables
gradient descent becomes a computationally expensive algorithm. We
will now discuss a variants of gradient descent that scales much
better.

The cost function of \(J\) is a bowl-shaped function (convex).
Stochastic gradient descent can be used both on linear regression and
 neural networks etc.

Gradient descent will  follow the gradient to the minimum.   If N is
large, computing the derivative is costly. 

The algorithm we have been using so far is called \idx{Batch gradient
  descent}.  If we use this dataset for really large datasets, we have
to stream all the data through memory for each iteration.  There just
isn't enough room in memory to keep it all there, and having done it
once, we then have to do it again for the next iteration.  It's fairly
costly.

The summarized batch gradient descent algorithm is based on the error function:

\[
  J_{\mbox{train}} = \frac{1}{2m}  \sum_{i=1}^{m}\parens{h_{\theta}(x^{(i)})  - y^{(i)}}^2
\]

And then to iterate this step:

\[
  \theta_j := \theta_j - \alpha\frac{1}{m}  \sum\parens{h_\theta(x^{(i)} - y^{(i)})x_j^{(i)}}
\]

for every \(j\in\braces{0,\ldots,n}\).  


In contrast the \idx{stochastic gradient descent} algorithm uses these
steps:

\[
\begin{array}{lcl}
\mbox{cost} (\theta, (x^{(i)}, y^{(i)})) &=& \frac{1}{2} h_\theta(x^{(i)}) - y^{(i)})^2\\
J_{\mbox{train}} &=& \frac{1}{m}\sum_{i=1}^{m} \mbox{cost} (\theta, (x^{(i)}, y^{(i)}))
\end{array}
\]

The cost of an error for a training set air is half the square of the
error produced by the hypothesis.  The cost function measures how the
hypothesis is doing on a single sample.   We can now define the
stochastic descent algorithm.

\begin{enumerate}
\item Randomly shuffle the dataset (standard reprocessing step).
\item repeat for  all the training examples for \(i \in  \brackets{1,\ldots,m} \):
\[
   \theta_j := \theta_j - \alpha \parens{h_{\theta}(x^{(i)} - y^{(i)}}
   \cdot x_j^{(i)}
\]

for \(j \in  \brackets{1,\ldots,n} \).   This thing makes use of the
fact that:

\[
\parens{h_{\theta}(x^{(i)} - y^{(i)}} =
\frac{\partial}{\partial\theta_j}\mbox{cost}\parens{\theta, (x^{(i)}),j^{(i)}}
\]

The algorithm is scanning through the training examples, and then it
will take a small ``gradient descent'' step and take another little
step through the parameter step and so on for each training example.

This view  of stochastic descent motivates why we want to shuffle the
data set, since we want it to be random :)  This will speed up the
convergence a bit.    


\xscreenshot{stochasticgradientdescent}{Convergence pattern for a
  stochastic gradient descent algorithm}

Another different is that we make progress in fitting tot he
parameters before we can make a little progress towards a global
minimum, we get progress for each training example we look at.

Batch gradient descent takes a reasonably straight line towards the
maximum, but stochastic gradient descent will have a somewhat more
erratic path towards the minimum.  It wil not necessarily have a
monotonous path towards the global minimum.  The path will look more
random.  Furthermore the algorithm doesn't really converge to the
global minimum and stay there, but this isn't a problem since it will
still end up in a region pretty close to the global minimum.  As a
practical matter, the result will be useful in most cases.  The inner
loop may have to be iterated one to ten times.  For really large
datasets it may be sufficient to have a single dataset through the
dataset.

Contrast this to  batch gradient desscent, where we would have to make
a lot of steps.   Implementing this algorithm will let us scale the
algorithm to much bigger datasets and thus get better performance.

\section{Mini batch gradient descent}


\xscreenshot{minibatchdescent}{minibatchdescent}

The \idx{Mini batch gradient descent} is somewhere in between the
\idx{stochastic gradient descent algorithm} and the \idx{batch
  gradient descent algorithm}.   In the mini-batch gradient descent we
use \(b\) samples in each iteration instead of just one as in
stochastic GD.

Typical choices for \(b\) is 10, and other typical choices may be
anywhere in the range \(2--100\).  The idea is that  we get (for
instance) ten examples \(x^{(i)}, y^{(i)}, \ldots, x^{(i+9)},
y^{(i+9)}\). Then we perform a gradient descent update (assuming b=10):

\begin{verbatim}

%There is a double superscript below, but
% what should it -really- be?
\[
   \theta_j = \theta_j - \alpha\frac{1}{10}  \sum_{k=i}^{}^{i+9}  \parens{h(x^{(k)}} - y^{(k)} \cdot x_j^{(k)}
\]
\end{verbatim}
The gradient term over the ten examples (the batch size) .

\item Iterate \(i := i +1 \).
\end{enumerate}

Again we don't need to scan through the entire training set before
making progress in modifying the parameters.

How about mini-batch gradient descent and not stochastic gradient
descent? And the answer is \idx{vectorization}.  By using appropriate
vectorization we can partially parallellize the gradient computation
and that can give a noticable improvement.   One problem is the
variable \(b\) that we now need to keep track of.

\section{Convergene of stochastic gradient descent}

We'll need to figure out how to manage convergence, and how to manage
the learning rate \(\alpha\).  In batch gradient descent we could
observe that the error was decreasing. We can't do that for stocastic
gradient descent since we need to scan through the entire training set
to compute the cost function.

So for stochastic  gradient descent we instead compute:

\[
\mbox{cost} (\theta, (x^{(i)}, y^{(i)})) = \frac{1}{2} h_\theta(x^{(i)}) - y^{(i)})^2\\
\]

{\em before} updating \(\theta\) using \(\parens{x^{(i)}, y^{(i)}}\).
As the algorithm is scanning through the training set, we compute how
well the hypothesis is doing on {\idx the training example we are
  working on}.

\screenshot{gradientdescentconvergence}{gradientdescentconvergence}

Then every  1000 iterations (or so) plot the \(\mbox{cost} (\theta,
(x^{(i)}, y^{(i)}))\) averaged over the 1000 examples.   This doesn't
cost much and we can keep track of how the algorithm is doing.  See
some examples in \ref{gradientdescentconvergence}.  With smaller
learning rates we may be able to get better results. Smaller learning
rates gives smaller oscillations, sometimes it doesn't matter much ;)
By increasing the number of iterations to smooth over the curve will
get smoother.  Sometimes it doesn't look like the algorithm is
learning.  It may be that averaging over larger samples will then show
a trend indicating that the algorithm is learning after all.  However,
it may also be flat, and that is an indication that the algorithm
isn't working.   If you se that that the error rate is learning, then
you are observing that the algorithm is \idx{diverging} and you should
use a smaller learning rate \(\alpha\).

The learning rate is usually held constant, but if you want the
stochastic gradient descent to actually converge on the global minmum,
you can slowly decrease the \(\alpha\) over time to help with
convergence, e,g.:

\[
      \alpha = \frac{\mbox{const1}}{\mbox{iterationNumber} + \mbox{const2}}
\]

The constantants are parameters you may have to play a bit with. They
may help, but they also increase the number of parameters you have to
mess with so it makes life harder.  This process is usually not used.

\section{Online learning}

Continous stream of data and we wish to have an algorithm to learn
from this.  Many large internet companies use this type of techniques,
often called \idx{online learning algorithm}s to analyze data.

Assume a case where you are running a site that sells packages.
Sometimes the uses wish to use a shipping service (y=1) and sometimes
not (y=0).   We wish to learn \(p(y=1 | x; \theta)\) to optimize
prize.  The \(x\) includes the price we ask for.  We can use logistic
regression (or any of the other algorithms, but let's use lr now).   

We will then repeat forever,  get an \((x,y)\) pair from the user.
The algorithm then update the parameters \(\theta\) using \((x,y)\):

\screenshot{onlinelearning}{Onlne learning.  Update using every data
  point as they arrive.}

\[
  \theta_j := \theta_j - \alpha (h_\theta(x) - y) \cdot x_j
\]

for \(j = 0, \ldots, n)\).

We are actually discarding the notion of having a learning set, we're
just processing the samples as they come in.   If you are running a
really high volume website this type of algorithm is really
useful. Data is essentially free, so we don't have to look for
training examples.  If we have fewer users it may be smarter to tuck
the dat away in a data set and run it.

The online approach has the advantage of being able to adopt to
changing user preferences.  The algorithm is following trends as they
happen.

Another appliction of online learnign is in product search.  Assume we
sell mobile phones.  We might get 100 phones that matches the search
``android phone with 1080p camera'', but we will return ten results.
We could have an algorithm to figure out which those ten phones should
be.  We may let clickthrough for users represent an \(y=1\) and
otherwise \(y=0\).  We can then use this data to estimate
\(p(y=1|x;\theta)\).  This is the problem of predicting ther
\idx{click through rate} (\idx{CTR}).  Since we can estimate the CTR,
we can now just select the ten phones with the highest estimated CTR.  

Every time we give the user ten choices, we actually get ten learning
examples.   We get them, run ten steps of gradient descent, and then
throw the dat away. 

There are other things we can do instead of estimating phones, we can
choose special offers to show the user, customized selection of news
articles, product recommendations etc.

Any of these problems could be formulated as a standard machine
learning problem with a training set, but if you get too much data,
there really isn't any point in saving away the training set, it's
better just to use the data on the fly.

The algorithm used in \idx{online learning}  is really very similar to the \idx{stochastic gradient
descent} algoritm


\section{Map reduce}

Some machine learning problems are just too big to run on a single
computer.    The map/reduce approach is very important even if we
won't spend much time on it than stochastic gradient descent.  The
reason less time is use is that because it's easier to explain.

\screenshot{mapreduce}{Map/Reduce: Divide and conquer :-)}

The map/reduce idea is to split the computational task into different
partitions.   Assume that we have four computers, we will split our
training set into four pieces.     The first machine will just use the
first quarter of the training set, and similarly for the rest of
them.     If we're working on a \idx{batch gradient descent}
algorithm, then our basic step is:

\[
    \theta_j := \theta_j - \alpha \frac{1}{400} \sum_{i=1}^{400} (h_\theta(x^{(i)} - y^{(i)}) x_j^{(i)}
\]

When we split this up we get temporary variables that summarize parts
of the big sum:

\[
    \mbox{temp}_j^{(1)} := \sum_{i=1}^{100} (h_\theta(x^{(i)} - y^{(i)}) x_j^{(i)}
\]

and then to do a similar things for the three other partitions.  Then
we collect the temporary variables and update like this:

\[
  \theta_j := \theta_j - \alpha\frac{1}{400} \sum_{i=1}^4 \mbox{temp}_j^{(i)}
\]

Then we do this separately for all the \(j=0, \ldots, n\).

This is exactly identical to batch gradient descent, but it's
parallell.

\screenshot{mapreduce2}{Dividing the learning task between multiple processors}

In practice we we get less than an n times speedup (latencies, etc.)

Many learning algorithms can be expressed as computing sums over
functions of the training set, and when they do they can be
implemented using map/reduce.
\screenshot{advancedmapreduce}{Map/Reduce requires some reformulations
in terms of partitioning of the data and operations (e.g. sums) over
those partitions.}

Assume that we wish to use an advanced optimization function
(LB.. etc.) calculating the cost function is expensive.  We can then
split the calculation of the error functions and partial derivatives
over many machines.

A single computer with multi-core CPUs can also be a nice place to use
map/reduce.  We can then split the job on diffent cores within the
same computer.  The advantage of thinking about map/reduce this way is
that we don't have to think much about network latency.

One last caveat on using multi-core machines.  Some libraries can
automatically paralleize over multiple cores, sometimes you can just
implement your algorithm in a vectorized fashion and the library will
take care of some of this automatically.   

Hadoop is a nice system for parallelizing algorithms to make them run
on really large datasets.

\chapter{Application example: Photo OCR}

\screenshot{photoocr}{Recognizing text in images is a hard task}

Big example.  Architecture and software engineering. Computer vision
and artificial data synthesis.   Photo  \idx{OCR} or \idx{Optical Character
  Recognition} lets a computer look for characters in pictures to
identify characters.  It dos so, and does it in several steps.  First
to identify where the characters are, and then to scan the texts and
come up with good interpretations.  While OCR for documents is
fairly well established, Photo OCR is still considered a hard machine
learning problem.

The pipeline is:
\screenshot{ocrpipeline}{A processing pipeline for processing text in images}

\begin{itemize}
\item Character recognition.
\item Character segmentation.
\item Character classification.
\end{itemize}


\xscreenshot{ocrpipelinegraphics}{ocrpipelinegraphics}

Some pipelines also has spelling correction, but we won't do much work
on that now.  A system like this is a \idx{machine learning
  pipeline}.   Pipelines like this are common in machine learning.

When designing a machine learning system designing the pipeline is
ofthen one of the most important decision that will be made.  If you
have a team of engineers working on this it can easily be one to five
engineers working on each of the subtasks.

In complex machine systems the concept of a pipeline is pretty
pervasive.

\subsection{Sliding windows}

\screenshot{textvspedestriandetection}{Text detection v.s. pedestarian
dectection.  Detecting pedestarians is a much simpler task since they
by and large all have the same orientation and aspect ration, whereas
text usually does not.}

The first component in the OCR pipeline we will look at a \idx{sliding
  window classifier}.  Detecting text is difficult among other things
since the text slots have different aspect ratios.  We'll start loking
at a simpler case, \idx{pedestrian detection} and then apply the
techniques we develop there in the text detction case.

In pedestrian detection we wish to find the individual pedestrians.
The aspect ratio of most pedestrians is about the same.   

\screenshot{pedestrianexamples}{Some examples of pedestarians, used as
input for machine learning.}

To build a pedestrian detector, we can choose a bunch of positive and
negative examples and a standard image size.   Several thousand
examples will probably be good. Some neural network, e.g. a neural
network can then be used to classify an image patch to detect if it
contains a pedestrian or not.   The algorithm is to extract a
rectangle of the correct aspect ratio and run it across the image to
search through.   The distance one steps is called \idx{stepsize} or
\idx{stride}, a stepsize of one gives the best result, but is
computaionally expensive.  A stride of eight or even higher will be
faster but have worse error rates.  Then then run larger image
patches, resizing it into the standard size (say 82x 36) and then run
that through the classifier.  ANd then do it at an even larger image.
This will detct squares with pedestrians in a window.


\screenshot{imagetosearchin}{Image to search for letters  in using a
  sliding windows algorithm.}
\screenshot{slidingwindowsoutput}{The output from the sliding windows
  character detector run on the picture in fig \ref{imagetosearchin}}


FInding text regions is harder, but it starts the same way.  Start
with a bunch of patches of images that contains text (and not) and
apply that to a  classifier algorithm.   As we run the sliding window
over the target window, we detect regions that may contain texts.
An input image is shown in fig. \ref{imagetosearchin} and the
corresponding output of the classifier is depicted in figure
\ref{slidingwindowsoutput}. The white patches on the result is to
indicate that the clasifiers thinks it might have found text.

\screenshot{classifiersummary}{A summary slide for the text detection
  using a sliding window and a expander algorithm}

We are not quite done yet, since we want to draw rectangles around the
texts.  To do this we first apply an \idx{expansion operator} on the output
from the classifier and get an image like the one in figure
\ref{expandedtext}.  The expansion option basically smear it a bit by
using  a rule that will color an expanded image white if it has a
white pixel nearer than five pixels.  We can then look at all the
white regions and apply a heuristic to expand rectangles with
reasonable aspect regions and then we're done.  This classifier
actualy misses some text that is hard to read (written against a
transparent window).

We can no use later stages in the pipeline to do character
segmentation.   We can again use a supervised learning alorithm.  The
first thing we should do is to lok for splits between characters.  We 

\screenshot{charactersegmentation}{Character segmentation using
  machine learning}

so we train a classifier to detect positive and negative examples, we
can then run this on the text detection system and figure out what are
individual characters and what is not.   We only need to slide the
classifier over a single row so it will be a one dimensional sliding
window.  This will give us a set of locations where we should split
the image into subimages containing single characters.

Finally we do character classification, and we know how to do that :-)
Standard a standard supervised learning algorithm to classify which
characters the images represent.

That is the photo OCR pipeline.

\section{Getting lots of data: Artificial data synthesis}

A low bias algorithm with a whole lot of training data is a really
good way of getting a good learning algorithm.  But where do we get
all the training data? If we have a lot of raw data that is good, but
if we don't we can cheat and make them ourselves using \idx{artificial
  data synthesis}.   It can't be applied to every problem, but if it
applies to the problem at hand, it is an easy way to get a huge data
set.   There are two basic techniques that are used, crating new data
from scratch, and to use a small training set and amplify that into a
larger training set.  We'll look at both of those ideas.


\screenshot{artificialsyntheticdata}{Real learning data v.s. synthetic
learning data}

Consider the \idx{artificial data synthesis idea}.  If we go out into
the world and collect a bunch of images we can use that.   Modern
computers often has a huge font library stored in it.  The are font
libraries that can be used to generate training examples by taking
characters from fonts and past it onto random backgrounds we may get
something that looks a lot like the set of images to the right in
figure \ref{artificialsyntheticdata}.  It is a bit of work to generate
this data, but it can certainly be done.  Use litt blurring, affine
(shearing/rotation/scaling) operations, and you get a training set.
If you do a sloppy work when you make the artificial data it won't
work very well.   Using this technique you have an essentially
unlimited supply of labeled data.  We just generate new data from
scratch.


\screenshot{warpedcharacters}{Warping characters to generate more
  learning input}

The other  approach is to synthesize data by introducing
distortions. In figure \ref{warpedcharacters} this is done by taking a
real image, and then warping it using sixteen different ``warpings''
:-) i.e. distorting filters.  Again, in order to do this for a
particular application this takes work.  The warpings must be
reasonable.  In speech recognition something entirely different will
be necessary :-)

\screenshot{choosingdistortions}{How to choose appropriate
  distortions.  Random noise is usually not a good choice.}

A word of warning: The distortions that are chosen must be ones that
are meaningful for the domain in question and ones you might expect to
see in the test set. Just adding random noise will usually not be very
helpful.

The process of artificial data synthesis is a bit of an art :-)

As always, make sure you have a low bias classifier before spending
the effort (plot learning curves).  Keep increasing the number fo
features until  you actually have a low bias classifier, and only then
you should start working with increasing the training set.

A question that is often asked is ``How much work would it be to get
ten times as much data as we currently have?'' It's a very good
question to ask.  Often the answer is ``it's not that hard'', and
often if you can get ten times as much data that is often a good way
to make the algorithm better.  Artificial data synthesis is one way
(both from scratch and distorting existing data).  Another way is to
collect/label data yourself.  It is useful to do the math on how hard
it is to get and label the examples yourself.  If it takes ten seconds
classify a datum, and you have say a thousand items then it will take
a ten thousand seconds, which is a bit about three hours of work.  It
is often surprising to see how little it can be to get a lot more data
and give the learning algorithm a huge boost in performance.

Finally we can ``crowdsource''.  We can even hire folks (e.g. through
the \idx{Amazon Mechanical Turk}) to do the labelling for us.   It can
be done.  It is often quite a bit of work to get high quality
labelling.

Remember to check the algorithm with learning curves, and figure out
how hard it is to get ten times as much learning data.

\section{Ceiling analysis}

\screenshot{ceilinganalysis}{Ceiling analysis: Figuring out where to
  get the most value for the effort expended on improving the
  processing pipeline.}

\idx{Ceiling analysis}: How to pick the part of the  pipeline to work
on next.   Consider the process in fig \ref{ceilinganalysis}, where should
you allocate resources?  Each of the boxes in figure
\ref{ceilinganalysis} represents somewhere we can add resources.
Again the idea of a \idx{single row number evaluation metric}.  Assume
that we have the character recognition metric for the OCR example.
If the system has 72 percent accuracy. We will now consider each
module and for every test example we'll provide the correct output.
We'll just manually tell it where the results are.  We will simulate
that we have a system with perfect accuracy.  It's easy, instead of
letting the algorithm do the work we do it for it.  We then do the
same type of thing for the other modules.  We'll give it both perfect
text detection an character detection. All the time we are looking at
the accuracy for the entire system.

We can now see which parts gives us the best upside.  In the example
in figure \ref{ceilinganalysis} the text detection system has a
potential upside of seventeen percent, the other parts much less.
This indicates that  working on text detection (in this example) is
the place where we can get the most bang for the buck.

We find the ``ceiling'' or the upper boundary for how much gain we can
get by making a module into a perfectly functioning component.

Let's consider another example, in this case \idx{face recognition
  from images}.   The example is artificial in the sense that this
isn't how face recognition is done in practice.

\screenshot{facerecogntionpipeline}{A face recognition pipeline}

\screenshot{facerecognitionceilinganalysis}{A ceiling analysis on the
  face recognition pipeline.}

We have a pipline consisting of a background remover, face detector,
eye segmentation, nose segmentation, mouth segmentation.  We then feed
all of this into a logistic regression classifier and that gives us a
label for the person.  This pipeline is probably too complicated for a
real face recognizer, but it's ok for a ceiling analysis example.
e can break down the system.

Cautionary story: There were a company  that let two engineers spend
a year and half to remove backgrounds, but it didn't make much
difference to the overall performance.  Doing a ceiling analysis
beforehand they could have prioiritized otherwise.

Our time as deveopers is very important. It's important to focus our
time on the component where we can make the most impact (gradient
descent:-).   Prof. Ng's experience has shown him that he shouldn't
trust his gut feeling very much.  It's better to do a ceiling analysis
to focus the effort.

%%
%% End of content
%%

\printindex
\end{document}

